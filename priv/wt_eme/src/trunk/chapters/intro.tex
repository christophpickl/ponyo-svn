
% XXXXXXXXXXXXXXXXXXXXXXXXXXXXXXXXXXXXXXXXXXXXXXXXXXXXXXXXXXXXXXXXXXXXXXXX

\renewcommand\chapterillustration{pushing_minimalistisch}
\chapter{Einf\"uhrung}

wer ans eingemachte will kann ruhig das folgende kapitel ueberspringen.

ein wenig theorie, ansonsten weiterlesen im eigenen kapitel theorie.

%\newpage

\section{Was ist Wing Chun?}
% =============================================================================

\subsection{Eine (innere) Kampfkunst?}

- internal vs external
- kampfkunst vs sport vs system/selbstverteidigung

\subsection{Verglichen mit Anderen}


\section{Schreibweisen}
% =============================================================================

[Wing Tschun]
Wing Tsun, Wing Chun, oder gar Ving Tjun, wie jetzt?!

siehe subsection verschiedene stile in section who is who.

! ab jetzt wird nur mehr die abkuerzung WT verwendet.

%You will notice that the name of the art is spelled many different ways (Wing Chun, Ving Tsun, Weng Chun, Wing Tsun, Wing Tsjun, Wing Tzun, Wing Tjun, Vinh Xuan, etc).  The original spelling was Wing Chun (and this is still the most common spelling used across all lineages).  This was later supposedly changed to Ving Tsun due to the fact that the British that were occupying Hong Kong were making fun of the initials WC, which refers to "Water Closet" or a bathroom.  The spelling was supposed suggested by Wong Shun Leung or Yip Man.  The Ving Tsun Athletic Association in Hong Kong, which was established in 1967, was the first to use this spelling.
%By using different spellings, organizations and schools generally want to indicate that they have a certain trademarked version of the system.  Wing Tsun was Leung Ting's way of trademarking his style of the art, and the Wing Tjun, Wing Tsjun, Wing Tzun's etc are all students of that lineage that broke away and still teach.  Weng Chun is an older lineage of the art taught in China, and Vinh Xuan is the Vietnamese lineage. "Traditional Wing Chun" is William Chung's trademarked version of the art.
%Although the spellings above reflect different Wing Chun lineages much of the spelling differences within the system itself are simply due to a lack of standardization in the romanization of the Chinese words.  The Chinese writing consists only of symbols. The different spellings have arisen by translating the consonance of the Chinese character to a Western phonetical alphabet. 