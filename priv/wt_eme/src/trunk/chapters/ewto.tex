
\newenvironment{WTGradSG}[1]{
	\subsection*{#1. Sch\"ulergrad}
	\includegraphics[width=1cm]{resources/graphics/graduierung/emblems/#1}
}{
}

\newenvironment{WTGradTG}[2]{
	\subsection*{#1. Techniker/Meistergrad}
	\ifthenelse{\equal{#2}{}}
	{}% #2 == empty
	{\includegraphics[width=1cm]{resources/graphics/graduierung/emblems/#2}}
%	\ifx&#2&%
%		ja
%	\else
%		nein
%	\fi
%	\includegraphics[width=1cm]{resources/graphics/graduierung/emblems/TG#1}
}{
}

% XXXXXXXXXXXXXXXXXXXXXXXXXXXXXXXXXXXXXXXXXXXXXXXXXXXXXXXXXXXXXXXXXXXXXXXX

\renewcommand\chapterillustration{pushing_minimalistisch}
\chapter{EWTO}

die ewto ist ein franchise unternehmen vom kernspecht.
sub von iwta, leung ting.
schweiz schembri, und wo es noch EWTO gibt.

\begin{figure}[h]
	\centering
	\includegraphics[width=8cm]{resources/graphics/logos/ewto}
	\caption{Das Logo der EWTO.}
\end{figure}

hat WT in unserer umgebung gross gemacht mit hilfe einer durchdachten kommerzialisierung.


\newpage

\section{Graduierung}
% =============================================================================

%bla bla

\begin{WTGradSG}{1}

BlitzDefence 1 und 2, SNT 1-3, Ritualkampf (Vorkampfphase, psychologie Zeugs), Kettenfaustst\"osse, (TODO KFST/VS???), LatSao 1??, SV1????, Trittabwehr am Boden.

\end{WTGradSG}
% *************************************************************************************
\begin{WTGradSG}{2}

SNT 4-8, BlitzDefence 3 und 4, Kettenfaustst\"osse mit Wendung bzw mit ZZ-Schritt, LatSao 2??, SV2???.

\end{WTGradSG}
% *************************************************************************************
\begin{WTGradSG}{3}

1/3 CK, Gleichzeitigkeiten (Drill), Dahn Chi Sao, Zeigen/Schubsen, LatSao 3??, SV3???.

\end{WTGradSG}
% *************************************************************************************
\begin{WTGradSG}{4}

2/3 CK, Schwinger (+Zyklus), Falten gegen Blocken nach oben/innen, PoonSao, LatSao 4??

\end{WTGradSG}
% *************************************************************************************
\begin{WTGradSG}{5}

Falten gegen Block nach unten/aussen, gegen Arm und Bein???, gegen starren Bong (Block)???, 3/3 CK, einf. Angr. A. Poon Sao??, Handfl\"achenstoss/VS usw.?, LatSao 5??

Ellbogenzyklus.

\end{WTGradSG}
% *************************************************************************************
\begin{WTGradSG}{6}

1. Sektion ChiSao 1/3 (Zug, bissi Djut-Kuen/LapDa), Glipschen, Messer Stich zum Bauch, Verteidigung Bodenlage, LatSao 6??.

\end{WTGradSG}
% *************************************************************************************
\begin{WTGradSG}{7}

1. Sektion ChiSao 2/3 (Innenpak), ReakTsun 1/LatSao??, Messer Stich von oben

\end{WTGradSG}
% *************************************************************************************
\begin{WTGradSG}{8}

1. Sektion ChiSao 3/3, ReakTsun 2/LatSao, Messer Schnitt Hals

\end{WTGradSG}
% *************************************************************************************
\begin{WTGradSG}{9}

ReakTsun 3/LatSao, Stockangriffe, LatSao "12 Angriffe" Teil 1

\end{WTGradSG}
% *************************************************************************************
\begin{WTGradSG}{10}

ReakTsun 4/LatSao, mehrere Angreifer, LatSao "12 Angriffe" Teil 2

\end{WTGradSG}
% *************************************************************************************
\begin{WTGradSG}{11}

ReakTsun 5/LatSao, sanfte Mittel, Verteidigung gegen Tritte, LatSao "12 Angriffe" Teil 3

\end{WTGradSG}
% *************************************************************************************
\begin{WTGradSG}{12}

ReakTsun 6/LatSao/+Schritt?, Stresstraining, LatSao "12 Angriffe" Teil 4, Bedrohung mit Schusswaffen

\end{WTGradSG}



% TODO ich glaub auf 1. TG gibts diese kondition/ausdauer pruefung...?!

% *************************************************************************************
\begin{WTGradTG}{1}{TG1}

\begin{itemize}
	\item 2., 3. und 4. Chi-Sao Sektion
	\item gegen Fingerzeigen, Schubsen, Greifen
	\item gegen wilden Schwinger
	\item 4 Blitze
	\item Drop-Punch nach vorne
	\item 24 ReakTsun-Routinen auch mit nicht-dominantem Arm
	\item 4 Phasen des Ausweichens und Konterns
	\item ReakTsun gegen Angriffe zum Kopf und gegen Tritte
	\item 4 Blitze z. Gegenangriff
	\item Folgen, F\"uhren, Befreien
	\item gegen Schlagwaffen
	\item AdrenalisaTsun
\end{itemize}

\end{WTGradTG}

% *************************************************************************************
\begin{WTGradTG}{2}{TG2}

\begin{itemize}
	\item 5., 6. und 7. Chi-Sao Sektion
	\item BiuDjie Form
	\item Abwehr gegen 4 Blitze
	\item vs Schubsen am K\"orper und mit Lan-Sao
	\item vs Schubsen aus beidseitigen Handgelenkgriff
	\item vs Greifen der Oberarme und Schubsen
	\item vs Schubsen gegen Schultern, Ellbogen
	\item Verteidigen und K\"ampfen in 2 Richtungen (WT90Grad)
	\item ReakTsun mit Stock
\end{itemize}
	
\end{WTGradTG}

% *************************************************************************************
\begin{WTGradTG}{3}{TG3}

\begin{itemize}
	\item Holzpuppenform (MYCF) 1-3
	\item vs Schubsen mit Schritt, vs Ziehen
	\item Messerabwehren, Messerkampf
	\item vom Chi-Sao ins Lat-Sao (Separation)
	\item Kraft borgen (Kollision)
	\item Klipp-Dreh-Tor-Modell
	\item Ver\"andern mit der Ver\"anderung des Gegners
	\item ReakTsun mit Stock
	\item BTCS 1-4
	\item Verteidigen und K\"ampfen 3 Richtungen (WT135Grad)
\end{itemize}
	
\end{WTGradTG}

% *************************************************************************************
\begin{WTGradTG}{4}{TG4}

\begin{itemize}
	\item MYCF 4-6 + Anwendung
	\item 1. Timing Moment Theorie und Praxis/Chi- und Lat-Sao
	\item 2. Timing Moment Theorie und Praxis/Chi- und Lat-Sao
	\item Bong-Sao, und wie gehts weiter?
	\item Geschmeidigkeitstraining
	\item Grosse Routine 1
	\item ReakTsun mit Stock
	\item Traditionelle Chi-Sao Formen wiederbesucht
	\item Verteidigung und Kampf in 4 Richtungen (WT180Grad)
\end{itemize}
	
\end{WTGradTG}

% *************************************************************************************
\begin{WTGradTG}{5}{}

\begin{itemize}
	\item Verteidigen und K\"ampfen in 5 Richtungen (WT225Grad)
	\item MYCF 7-8 + Anwendung
	\item ChiGoerk 1-2 + Anwendung
	\item \ldots
\end{itemize}

\end{WTGradTG}

% *************************************************************************************
\begin{WTGradTG}{6}{}

\begin{itemize}
	\item Langstockform und ChiKwan
	\item Verteidigen und K\"ampfen in 6 Richtungen (WT270Grad)
	\item \ldots
\end{itemize}

\end{WTGradTG}

% *************************************************************************************
\begin{WTGradTG}{7}{PG3}

\begin{itemize}
	\item BCD 1-4
	\item Verteidigen und K\"ampfen in 7 Richtungen (WT315Grad)
	\item Geschmeidigkeit 1/2
	\item Gleichgewicht 1/2
	\item Koerpereinheit 1/2
	\item Timing 1/2
	\item Provokation 1/2
	\item Tastsinn 1/2
	\item \ldots
\end{itemize}

\end{WTGradTG}

% *************************************************************************************
\begin{WTGradTG}{8}{}

\begin{itemize}
	\item BCD 5-8
	\item Verteidigen und K\"ampfen in 8 Richtungen (WT360Grad)
	\item Geschmeidigkeit 2/2
	\item Gleichgewicht 2/2
	\item Koerpereinheit 2/2
	\item Timing 2/2
	\item Provokation 2/2
	\item Tastsinn 2/2
	\item \ldots
\end{itemize}

\end{WTGradTG}



\section{Akademie Wien}
% =============================================================================

\subsection{Stundenplan}
% *************************************************************************************

bla bla.

\begin{table}[h]
	\centering
	\begin{tabular}{l|l|l|l|l|l|l}
		\multicolumn{1}{c|}{\textbf{Lehrer}} &
		\multicolumn{1}{c|}{\textbf{Montag}} &
		\multicolumn{1}{c|}{\textbf{Dienstag}} &
		\multicolumn{1}{c|}{\textbf{Mittwoch}} &
		\multicolumn{1}{c|}{\textbf{Donnerstag}} &
		\multicolumn{1}{c|}{\textbf{Freitag\footnote{Am Freitag finden die Abendeinheiten eine Stunde fr\"uher statt, also um 17:00 und 19:00 Uhr.}}} &
		\multicolumn{1}{c}{\textbf{Samstag}} \\ \hline \hline
		10:00 - 11:30 & Sifu Matthias & & & &
					Christian &
					(frueher Escrima) \\ \hline
		13:00 - 14:30 & & & & & &
					\WTcell{Stefan,\\(fr\"uher Ado)} \\ \hline
		15:00 - 16:30 & & & 
					Kosti & & &
					\WTcell{Stefan,\\(fr\"uher Ado)} \\ \hline
		16:45 - 17:45 & & &
					\WTcell{Kosti bzw.\\Gerhard} & & & \\ \hline
		18:00 - 19:30 &	Sifu Matthias &
					\WTcell{Thomas,\\Matthias} &
					Gerhard &
					? Thomas &
					\WTcell{Ernst,\\Mario} & \\ \hline
		20:00 - 21:30 & Herbert &
					Robert &
					Franz &
					Martin &
					Gernot? & \\
	\end{tabular}
	\caption{Der EWTO Stundenplan der Akademie Wien}
\end{table}


%\newpage
\subsection{Lehrer}
% *************************************************************************************
\newcommand{\WTEWTOLehrer}[3]{
	\begin{tabular}{ll}
		\WTXEWTOLehrerGraphic{#1} & \WTXEWTOLehrerText{\textbf{{\LARGE #2}}\\ #3} \\
	\end{tabular} \\
}
\newcommand{\WTXEWTOLehrerGraphic}[1]{\WTXCommonImageTop{\includegraphics[width=55pt]{resources/images/ewtolehrer/#1}}}
\newcommand{\WTXEWTOLehrerText}[1]{\raisebox{-0.8cm}{\parbox{0.7\linewidth}{#1}}}

Es folgt ein kleiner Text. Es folgt ein kleiner Text. Es folgt ein kleiner Text. Es folgt ein kleiner Text. Es folgt ein kleiner Text. Es folgt ein kleiner Text. Es folgt ein kleiner Text. Es folgt ein kleiner Text.

\begin{flushleft}
\WTEWTOLehrer{ademir}{Ademir}{sehr praxisorientiert, kampfgeist, gute uebungen fuer skills}
%\WTEWTOLehrer{andreas}{Andreas}{chi kung}
\WTEWTOLehrer{christian}{Christian}{schnelligkeit, kampfgeist}
\WTEWTOLehrer{ernst}{Ernst}{gutes wt, ehrlicher typ}
\WTEWTOLehrer{gerhard}{Gerhard}{schlagtraining, kombos}
\WTEWTOLehrer{gernot}{Gernot}{echte effektivitaet, sehr praxisorientiert, starker kampfgeist}
\WTEWTOLehrer{herbert}{Herbert}{xxx}
%\WTEWTOLehrer{james}{James}{frueher mit dem 2. thomas am samstag}
\WTEWTOLehrer{kostadin}{Kosti}{einer der juengeren lehrer}
\WTEWTOLehrer{mario}{Mario}{praezision pur}
\WTEWTOLehrer{martin}{Martin}{rescher stil, praxisorientiert, vorkampf-fertigkeiten}
%\WTEWTOLehrer{martina}{Martina}{frauenbehauptung}
\WTEWTOLehrer{matthias_gold}{Sifu Matthias}{Schulleiter, weichheit, timing}
% TODO \WTEWTOLehrer{matthias}{Matthias}{der lange, mit grauen haare}
\WTEWTOLehrer{robert}{Robert Sr.}{gut erklaert, Bodenkampf}
\WTEWTOLehrer{robert_junior}{Robert Jr.}{der (vom alter) juengste lehrer}
\WTEWTOLehrer{thomas}{Thomas}{gute beispiele, immter interessant, 12 angriffe}
%\WTEWTOLehrer{thomas_oarge}{}{frueher mit dem james am samstag}
%TODO \WTEWTOLehrer{stefan}{Stefan}{ganz neu}
\end{flushleft}
