
\newenvironment{WTTechnik}[1]
	{
		\subsection{#1}
	}
	{}
	

\renewcommand\chapterillustration{geometry_triangle}
\chapter{Techniken}


\WTCommonCite{Unbekannt}{Du musst den falschen Weg gehen, um das richtige Ziel zu finden.}

Es gibt eigentliche keine Techniken im WT, einzig die Prinzipien bestimmen die Entscheidungen im Moment. Trotzdem gibt es sie, um dem Lernenden eine Art Kr\"ucke zur Verf\"ugung zu stellen, da es sonst schwer w\"are ein so abstraktes Konzept ohne konkrete Beispiele zu vermitteln.


\newpage



% ==========================================================================================
\section{Sieben Grundpositionen}
% ==========================================================================================


\begin{WTTechnik}{Wu-Sao}
	Asdf
% wu sao		schuetzende hand
	%[protective / hand, arm] - "praying" palm, rear guard hand, guarding hand
\end{WTTechnik}

%%%%%%%%%%%%%%%%%%%%%%%%%%%%%%%%%%%%%%%%%%%%%%%%

\begin{WTTechnik}{Man-Sao}
	Asdf
%man sao	suchende hand
	%[ask, investigate / hand, arm] - inquisitive hand, forward guard hand
\end{WTTechnik}

%%%%%%%%%%%%%%%%%%%%%%%%%%%%%%%%%%%%%%%%%%%%%%%%
\begin{WTTechnik}{Tan-Sao}
	\WTGaleryImageTwo{positionen/tan_sao-front}{positionen/tan_sao-seite}{Tan-Sao Demonstration frontal und seitlich}
	Der Tan-Sao, auch \quoteit{Handfl\"ache nach oben Hand} genannt, ist eine absorbierende Technik.
	%Bei der Form Ellbogen innen, sonst Situation anpassen. Flache Hand. Daumen dran. Handfl\"ache parallel zum Boden (oder auch Verl\"angerung zum Unterarm).
	% TODO es gibt den tan sao in verschiedenen variationen: ... siehe buch ...
% tan sao		handflaeche oben hand; absorbing/dispersing hand
	%open, spread out / hand, arm] - palm up arm, begging palm, a high outside block
\end{WTTechnik}

%%%%%%%%%%%%%%%%%%%%%%%%%%%%%%%%%%%%%%%%%%%%%%%%

\begin{WTTechnik}{Bong-Sao}
	Asdf
% bong sao	schwing/fluegel arm
	%[wing; arm] mid to high inside redirection block 
\end{WTTechnik}

%%%%%%%%%%%%%%%%%%%%%%%%%%%%%%%%%%%%%%%%%%%%%%%%

\begin{WTTechnik}{Lan-Sao}
	Asdf
% lan sao		riegelarm; bar/barring
	%[hinder / hand, arm] - bar arm
\end{WTTechnik}

%%%%%%%%%%%%%%%%%%%%%%%%%%%%%%%%%%%%%%%%%%%%%%%%

\begin{WTTechnik}{Fook-Sao}
	Asdf
%fook/fok sao	bruecken arm, ~hook sao; controlling/cover hand
	% [supress / arm] - bridge hand, bridge-on arm, monkey hand; a middle inside block.; Fook [supress, strike, lay down, hide] 
\end{WTTechnik}

%%%%%%%%%%%%%%%%%%%%%%%%%%%%%%%%%%%%%%%%%%%%%%%%

\begin{WTTechnik}{Kau-Sao}
	Asdf
% kao/kau sao	~klapp sao; detaining hand, Circling-Arm
	% circling block
\end{WTTechnik}


% ==========================================================================================
\section{Vierzehn Grundbewegungen}
% ==========================================================================================


\begin{WTTechnik}{Huen-Sao}
	Asdf
% huen (hyoon) sao	zirkelnde hand
	%  [circle / hand, arm] - circling wrist redirection block
\end{WTTechnik}

%%%%%%%%%%%%%%%%%%%%%%%%%%%%%%%%%%%%%%%%%%%%%%%%

\begin{WTTechnik}{Gan-Sao}
	Asdf
% gan/gaun sao	schneidende hand; cultivating arm
	%[poling?, as in poling a boat / hand, arm] - low block, splitting block, cultivating hand, dividing hand; a low outside block
\end{WTTechnik}

%%%%%%%%%%%%%%%%%%%%%%%%%%%%%%%%%%%%%%%%%%%%%%%%

\begin{WTTechnik}{Gam-Sao}
	Asdf
% gam/gum sao	pressing hand
	%[pin, press / hand, arm].
\end{WTTechnik}

%%%%%%%%%%%%%%%%%%%%%%%%%%%%%%%%%%%%%%%%%%%%%%%%

\begin{WTTechnik}{Pak-Sao}
	Asdf
% pak (paak) sao	klatschende hand; slapping
	% [touch, clap, pat / hand, arm] - slap block, prayer palm block, open hand block
\end{WTTechnik}

%%%%%%%%%%%%%%%%%%%%%%%%%%%%%%%%%%%%%%%%%%%%%%%%

\begin{WTTechnik}{Lap-Sao}
	Asdf
% lap sao		ziehende hand
	% [break, crush / hand, arm] - deflecting arm, warding-off hands, controlling hands, grabbing hand
\end{WTTechnik}

%%%%%%%%%%%%%%%%%%%%%%%%%%%%%%%%%%%%%%%%%%%%%%%%

\begin{WTTechnik}{Djam-Sao}
	Asdf
% jum/cham/djam/djum sao	sinkender ellbogen
	% [sink; hand, arm] sinking block
\end{WTTechnik}

%%%%%%%%%%%%%%%%%%%%%%%%%%%%%%%%%%%%%%%%%%%%%%%%

\begin{WTTechnik}{Djat-Sao}
	Asdf
% jut/djat sao		abwuergende (und schlagende) Schockhand, choking hand
	%[block, stop up / hand] - jerk hand, snake hand, shock hand.
	% kurze und kraeftige bewegung
\end{WTTechnik}

%%%%%%%%%%%%%%%%%%%%%%%%%%%%%%%%%%%%%%%%%%%%%%%%

\begin{WTTechnik}{Tok-Sao}
	Asdf
% tok sao		hebende hand; lifting hand
	% [? / hand, arm] - (double) upward palm, lifting hand technique
\end{WTTechnik}

%%%%%%%%%%%%%%%%%%%%%%%%%%%%%%%%%%%%%%%%%%%%%%%%

\begin{WTTechnik}{Kwan-Sao}
	Asdf
%kwan/kwun/quan sao	rotierender arm
%Kwan sao [arrange, plait, beat firm / hand, arm] - rotating arms; trapping hand; (high) b�ng & t�n combination movement; also used for block with outer edge of wrist, palm up; similar to t�n s�o but goes in opposite direction to close off center.
\end{WTTechnik}

%%%%%%%%%%%%%%%%%%%%%%%%%%%%%%%%%%%%%%%%%%%%%%%%

\begin{WTTechnik}{Kwat-Sao}
	Asdf
% kwat/kwut/gwat/quat sao		wischender arm; auch "Bong-Sao-Killer"
	% Sweeping Arm
\end{WTTechnik}

%%%%%%%%%%%%%%%%%%%%%%%%%%%%%%%%%%%%%%%%%%%%%%%%

\begin{WTTechnik}{Biu-Tse-Sao}
	Asdf
% biu sao / biu tse (Byoo jee) sao		darting hand
	% [a dart, go forward, thrust, fly, poke, point; finger(s), hand, arm, point at; hand, arm] darting fingers, thrusting fingers (hand or arm), finger jab 
\end{WTTechnik}

%%%%%%%%%%%%%%%%%%%%%%%%%%%%%%%%%%%%%%%%%%%%%%%%

\begin{WTTechnik}{Tuet-Sao}
	Asdf
% tuet sao: der kratzende Arm, die Hand "verlieren"
	% 3 Meanings: strip, disrobe, disengage, escape. 
\end{WTTechnik}

%%%%%%%%%%%%%%%%%%%%%%%%%%%%%%%%%%%%%%%%%%%%%%%%

\begin{WTTechnik}{Lau-Sao}
	Asdf
% Lao/lau sao		schoepfende hand
	% scooping arm, slippery hand
\end{WTTechnik}

%%%%%%%%%%%%%%%%%%%%%%%%%%%%%%%%%%%%%%%%%%%%%%%%

\begin{WTTechnik}{Tai-Sao}
	Asdf
% tai sao			hebende arme
	% so wie glockenschlag; am ende 4. satz nach lange bruecke
\end{WTTechnik}

% ? tie sao		uplifting hand
% ? jip sao		receiving hand



% ==========================================================================================
%\section{Der Handfl\"achenstoss}
% ==========================================================================================


%Djark-Cheung - Seitliche Handfl\"ache (bis auf Schulterbreite; satz 3+5);
%?Ching-Cheung - Gerade Handfl\"ache (satz 3);
%Wang-Cheung - Liegende Handfl\"ache (satz 5);
%Dai-Cheung - Liegende Handfl\"ache zu den kurzen Rippen (satz 6)
%Ong-Cheung - Verkehrte Handfl\"ache (satz 7)


% ==========================================================================================
%\section{Der Fauststoss}
% ==========================================================================================

%Anatomie, etc.

