\newenvironment{WTPosition}[2]
	{
		\subsubsection{#1}
		\textit{#2}
		\\
	}
	{}
	

\renewcommand\chapterillustration{six-computers-chips-circuits}
\chapter{Positionen \& Bewegungen}


\WTCommonCite{Lao-Tse}{Ein ur cooles \\ Zitat vom Lao Tse.\\ Ein ur cooles Zitat vom Lao Tse. Ein ur cooles Zitat vom Lao Tse. Ein ur cooles Zitat vom Lao Tse. Ein ur cooles Zitat vom Lao Tse. Ein ur cooles Zitat vom Lao Tse. Ein ur cooles Zitat vom Lao Tse. Ein ur cooles Zitat vom Lao Tse.}

Bla bla. Wozu.


\newpage



\section{Sechs Grundpositionen}

\begin{WTPosition}{Tan-Sao}{Handfl\"ache oben Hand}
	
	Ist eine absorbierende Hand. Bei der Form Ellbogen innen, sonst Situation anpassen. Flache Hand. Daumen dran. Handfl\"ache parallel zum Boden (oder auch Verl\"angerung zum Unterarm).
	
	%\WTGaleryImage{positionen/tan_sao-front}{9cm}{Tan-Sao Demonstration frontal}
	\WTGaleryImageTwo{positionen/tan_sao-front}{positionen/tan_sao-seite}{Tan-Sao Demonstration frontal und seitlich}
\end{WTPosition}


\section{Zw\"olf Grundbewegungen}


% bong sao	schwing/fluegel arm
	%[wing; arm] mid to high inside redirection block 
%fook/fok sao	bruecken arm, ~hook sao; controlling/cover hand
	% [supress / arm] - bridge hand, bridge-on arm, monkey hand; a middle inside block.; Fook [supress, strike, lay down, hide] 
%man sao	suchende hand
	%[ask, investigate / hand, arm] - inquisitive hand, forward guard hand
% wu sao		schuetzende hand
	%[protective / hand, arm] - "praying" palm, rear guard hand, guarding hand
% tan sao		handflaeche oben hand; absorbing/dispersing hand
	%open, spread out / hand, arm] - palm up arm, begging palm, a high outside block
% kau sao	~klapp sao; detaining hand
	% circling block
%
% jum/cham sao	sinkender ellbogen
	% [sink; hand, arm] sinking block
% gan/gaun sao	schneidende hand; cultivating arm
	%[poling?, as in poling a boat / hand, arm] - low block, splitting block, cultivating hand, dividing hand; a low outside block
% jut/djat sao		choking hand
	%[block, stop up / hand] - jerk hand, snake hand, shock hand.
% huen (hyoon) sao	zirkelnde hand
	%  [circle / hand, arm] - circling wrist redirection block
% lap sao		ziehende hand
	% [break, crush / hand, arm] - deflecting arm, warding-off hands, controlling hands, grabbing hand
% pak (paak) sao	klatschende hand; slapping
	% [touch, clap, pat / hand, arm] - slap block, prayer palm block, open hand block
% tok sao		hebende hand; lifting hand
	% [? / hand, arm] - (double) upward palm, lifting hand technique
% lan sao		riegelarm; bar/barring
	%[hinder / hand, arm] - bar arm
% ? tie sao		uplifting hand
% ? jip sao		receiving hand
% gam/gum sao	pressing hand
	%[pin, press / hand, arm].
% biu sao / biu tse (Byoo jee) sao		darting hand
	% [a dart, go forward, thrust, fly, poke, point; finger(s), hand, arm, point at; hand, arm] darting fingers, thrusting fingers (hand or arm), finger jab 
% Lao sao		schoepfende hand
	% scooping arm, slippery hand


%kwan (quan) sao	rotierende hand
%Kwan sao [arrange, plait, beat firm / hand, arm] - rotating arms; trapping hand; (high) b�ng & t�n combination movement; also used for block with outer edge of wrist, palm up; similar to t�n s�o but goes in opposite direction to close off center.

\subsection{Handfl\"achenst\"o{\ss}e}

Djark-Cheung - Seitliche Handfl\"ache (bis auf Schulterbreite; satz 3+5);
?Ching-Cheung - Gerade Handfl\"ache (satz 3);
Wang-Cheung - Liegende Handfl\"ache (satz 5);
Dai-Cheung - Liegende Handfl\"ache zu den kurzen Rippen (satz 6)
Ong-Cheung - Verkehrte Handfl\"ache (satz 7)