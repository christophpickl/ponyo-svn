
% XXXXXXXXXXXXXXXXXXXXXXXXXXXXXXXXXXXXXXXXXXXXXXXXXXXXXXXXXXXXXXXXXXXXXXXX

\renewcommand\chapterillustration{pushing_minimalistisch}
\chapter{Chi-Sao}

was ist es: trainingshilfsmittel (kein kampf), klebende haende, cling arms, (mit chi arbeiten), taktiles, reflexe, variante des sparrings

%\newpage

\section{Vorbreitungen f\"ur's Chi-Sao}
% =============================================================================

\subsection{Doc-Sao}

aka \quoteit{freeze frame}. varianten: immer eins, dann mit drei angriffen, dann durchgehend angreifen mit unterbrechungen/angriffen zwischendurch; dann fluessig und man macht freies ;)

\subsection{Armpositionen}

anmerkung: jetzt muss ich dem leser etwas vorgreifen: chi-sao mit beiden armen, wie man da die arme haelt.

fok+tan, vs fok+bong

\subsection{Dan-Chi-Sao}

einarmig, vorstufe fuer anfaenger

\subsubsection{Sequenzen}

handflaechenstoss dann djam, auf bong zurueckdraengen, etc.

\subsection{Freies Chi-Sao}

ohne kraft, luecken suchen und luecken wieder schliessen, macht richtig spass mit einem geeigneten partner


\section{Sektionen}
% =============================================================================

partnerformen, wie beim tanz eine choreographie, man muss zusammen spielen und nicht gegeneinander kaempfen

\subsection{Erste Sektion}

auf drei drittel
zug, pak, ???

\subsection{Sektion Zwei bis Sieben}

???

\section{Spielereien}
% =============================================================================

\subsection{Freies Chi-Sao}


