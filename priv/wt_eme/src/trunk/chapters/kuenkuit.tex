
\renewcommand\chapterillustration{buddha_statue}
\chapter{Kuen Kuit}
%These are short, often sing-song, sayings or rhymes that indicate principles, or strategies, or even particular responses

%\WTCommonCite{Lao-Tse}{Ein ur cooles \\ Zitat vom Lao Tse.\\ Ein ur cooles Zitat vom Lao Tse. Ein ur cooles Zitat vom Lao Tse. Ein ur cooles Zitat vom Lao Tse. Ein ur cooles Zitat vom Lao Tse. Ein ur cooles Zitat vom Lao Tse. Ein ur cooles Zitat vom Lao Tse. Ein ur cooles Zitat vom Lao Tse.}
%\newpage


\section{Prinzipien}


\subsection{Kampfprinzipien}

\begin{enumerate}
	\item stoss vor
%	    1. Ist der Weg frei, stoss vor!
%    St�ndiger Kraftfluss nach vorne, auf das Ziel zu, denn Gegenangriff ist die beste Verteidigung. Sobald die eigene Sicherheitsdistanz vom Kontrahenden unterschritten wird, folgt der Gegenangriff.
	\item bleib kleben
 %  2. Wenn der Weg nicht frei ist, bleib kleben!
%    Dranbleiben: Wird die Verteidigungsaktion vom Gegner gestoppt, beh�lt WingTsun den Kontakt und somit die Kontrolle bei, um einen neuen Angriffspunkt zu finden (anstatt sich zur�ckzuziehen).
	\item gib nach
%    3. Wenn die Kraft des Gegners gr�sser ist, gib nach!
 %   Der Kl�gere gibt nach: �bt der Gegner nun einen verst�rkten Druck aus, um die Kontrolle zu brechen, gibt WingTsun weich nach. Man l�sst den Angriff ins Leere laufen, ohne dabei die Kontrolle aufzugeben (anstatt sich dagegenzustemmen mit unzureichenden Kr�ften).
	\item folge
%    4. Zieht der Gegner sich zur�ck, folge!
%    Zieht sich der Angreifer zur�ck, z.B. um zum neuen Schlag auszuholen, dringt WingTsun sofort nach vorne nach, fliesst automatisch wie Wasser in jede sich ergebende L�cke. Dies ist eine Konsequenz aus dem Aus�ben des 1. Prinzips, dem st�ndigen Vorw�rtsdrang. 
\end{enumerate}

\subsection{Kraftprinzipien}

\subsection{Prinzip der Gleichzeitigkeit}

\subsection{Blick folgt der Technik}


\subsection{TODO veraendern augrund der veraenderung des gegners}

sich anpassen, wie wasser, flexibel auf situation reagieren, ohne ueber eine gespeicherte technik nachzudenken, im moment sein

\subsection{Reiz\"uberflutung}


\section{Konzepte}

\subsection{Zentrallinie}

\subsection{Keilform}

\subsection{Verwurzelung}

\subsection{Federspannung}

\subsection{Peitscheneffekt}