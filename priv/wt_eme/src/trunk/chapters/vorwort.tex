\section*{Vorwort}
\addcontentsline{toc}{section}{Vorwort}

% grund warum geschrieben, zielsetzung

Die folgenden Seiten beinhalten Informationen f\"ur \textbf{Wing Chun Anf\"anger} und stellt eine Art \textbf{Kompendium} dar, eine strukturierte Zusammenfassung die als Nachschlagewerk verwendet werden kann. Es kann auch mehr als ein Notiz- oder Tagebuch mit gesammelten Erfahrungen angesehen werden. Man kann es linear lesen, oder auch dient es als Nachschlagewerk f\"ur einzelne Bereiche, wie z.B. Formen, die beliebten Drills und Zyklen, Pr\"ufungsstoff, Chinesische Fachausdr\"ucke, Theorie, und mehr.

% zielgruppe (EWTO) anfaenger

Wie schon erw\"ahnt richtet sich das Buch an Anf\"anger, darum ist es auch eher ein 1x1 als ein \quoteit{Profibuch}, wobei nicht ausgeschlossen ist dass nicht auch erfahrene Wing Chun Praktizierende noch die ein oder andere Wissens\"ucke f\"ullen k\"onnen. Weiters sei angemerkt dass vor allem Sch\"uler der \textbf{EWTO} (Europ\"aische WingTsun Organisation) hier viele Hilfreiche Informationen vorfinden werden, da sogar auf Dinge eingegangen wird wie Lehrer und der Stundenplan der Akademie Wien, als auch klarerweise das Lehrstoff der dort existierenden (Sch\"uler-)Programme.

% keinen anspruch auf richtigkeit/vollstaendigkeit

% TODO MINOR kein wahrer meister wuerde buch schreiben, darum gibts keine ``meisterhaften buecher''
% TODO MINOR WT lern man nur durchs tun, nicht durchs lesen (ist jedoch gute ergaenzung); bsp mit radfahren lernen

Ich m\"ochte mich schon im vorhinein entschuldigen f\"ur diverse Unrichtigkeiten und Unvollst\"andigkeiten, ich besitze auch nur einen Anf\"angergeist, und somit sind die hier niedergeschriebenen Erfahrungen und Meinungen nicht als \textit{die Wahrheit} zu verstehen. Auch die Tatsache dass es viele, sch\"one unterschiedliche Stile und Ansichten gibt, erschwert das Unterfangen eine Zusammenfassung \"uber eine Kampfkunst wie Wing Chun zu schreiben. Gerade Wing Chun, dass bekanntlich \textbf{Prinzip-basierend} ist (im Gegensatz zum Erlernen von Techniken), und somit viel Spielraum offen l\"asst f\"ur Interpretationen.

% danksagung

Ein vom Herzen kommendes Danke an all die netten Menschen mit denen ich trainieren durfte und auch noch in Zukunft trainieren werde, mit denen ich gemeinsam durch Zeiten harten Arbeitens lernen und mich weiterentwickeln konnte und natuerlich auch an all meine eigenen Lehrer die mir meinen Weg aufzeigen konnten, sowie Personen die mir Sch\"uler waren denen ich hoffentlich mindestens soviel weitergeben konnte wie mir weitergegeben wurde. Respekt und Aufrichtigkeit, One Love
