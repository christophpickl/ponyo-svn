
\newenvironment{WTSatzTeil}[2]
	{\paragraph{#1} (\textit{#2})}
	{}

\newenvironment{WTSatz}[1]
	{\WTGaleryResetSlideshowCounter \subsubsection{#1}}
	{}

\def\WTXFormen_EingangsGraphics#1{\includegraphics[width=2.2cm]{resources/images/eingangsform/#1}}


\section{Formen}

Die Formen setzen sich aus einer Reihe von vordefinierten Bewegungsabl\"aufen zusammen, die von jedem WT-Praktizierenden regelm\"a{\ss}ig, aufmerksam und bewusst durchgef\"uhrt werden sollte. Die Art der Ausf\"uhrung hierbei kann gerne variieren: mal schnell/langsam, mal hart/weich, mal auf dem einen/anderen Bein, mit nur einer Hand, etc. Auf jeden Fall sollte nicht durch Hetzerei die Bewegung \textit{verwischt} werden, sondern jeder Punkt noch durchlaufen werden.

Nach l\"angerem \"Uben schleifen sich die Bewegungen ein und sind sofort f\"ur einen abrufbar. Dies gibt einem die M\"oglichkeit sich eine Art Werkzeugkasten zu bauen, aus dem man sich dann bei Bedarf bedient um eine konkrete Technik umzusetzen.

\subsection{Begin und Schluss}
%%%%%%%%%%%%%%%%%%%%%%%%%%%%%%%%%%%%%%%%%%
%%%%%%%%%%%%%%%%%%%%%%%%%%%%%%%%%%%%%%%%%%

Die Anfangs- und Endbewegung ist bei jeder Form immer die gleiche und setzt sich aus folgenden Bewegungen zusammen:

\begin{figure}[htbp]
	\centering
	\begin{tabular}{ccccc}
		\WTXFormen_EingangsGraphics{arm1} & \WTXFormen_EingangsGraphics{arm2} & \WTXFormen_EingangsGraphics{arm3} & \WTXFormen_EingangsGraphics{arm3} & \WTXFormen_EingangsGraphics{arm3} \\
		\WTXFormen_EingangsGraphics{bein1} & \WTXFormen_EingangsGraphics{bein2} & \WTXFormen_EingangsGraphics{bein3} & \WTXFormen_EingangsGraphics{bein4} & \WTXFormen_EingangsGraphics{bein5} \\
		a) & b) & c) & d) & e) \\
	\end{tabular}
\end{figure}

Die \textbf{Eingangsform} geht von einem lockeren Stand aus, und f\"uhrt in den sogenannten IRAS\index{IRAS}-Stand, was soviel bedeutet wie: \textit{\textbf{I}nternally \textbf{R}otated \textbf{A}dduction \textbf{S}tance}. Hierbei werden die Knie zueinandergepresst wobei die Oberschenkelmuskulator (die Adduktoren; v. lat.: \textit{adducere}, hinf\"uhren, hinziehen) durchgehend gespannt und somit trainiert werden. Dieser Stand versteht sich nur als Trainingsstand zur Verbesserung der Beinmuskulator und ist nicht gedacht zum Einsatz im Ernstfall.

\begin{WTalphenum}
	\item Ausgangspunkt ist ein aufrecher Stand, der Blick ist nach vorne gerichtet, der Kopf sitzt weiter hinten, die Arme h\"angen locker auf der Seite hinunter und die F\"usse sind gschlossen.
	\item Nun werden beide H\"ande der Seite entlang brusthoch angezogen.
	\item Man setzt sich in die Knie.
	\item Die Zehenspitzen drehen so weit wie m\"oglich auf den Fersen auseinander.
	\item Danach folgen die Fersen und werden \"uber die Fussmitte aufgemacht.
\end{WTalphenum}

\WTCommonNoob{Man dreht gerne auf den Zehen auseinander was zu einem falschen Stand f\"uhrt.}

\WTGaleryImageTwo{eingangsform/fertig-front}{eingangsform/fertig-seite}{Fertiger IRAS Stand f\"ur die Form}

Am Ende gibt es dann eine kurzgehaltene \textbf{Ausgangsform} zur\"uck in einen lockeren Stand.

\begin{WTalphenum}
	\item Ausgangspunkt ist der volle IRAS Stand.
	\item Als erstes wird der rechte Fuss gerade gestellt, sodass Zehen und Ferse eine gerade Linie nach vorne bilden.
	\item Nun stellt (noch immer gehockt) der linke Fuss zum rechten parallel Fuss mit kleinem Abstand zu.
	\item Erst jetzt werden die Beine aus der Hocke gestreckt.
	\item Und zu letzt werden die Arme an der Seite locker hinuntergef\"uhrt.
\end{WTalphenum}


\subsection{Siu-Nim-Tao}
%%%%%%%%%%%%%%%%%%%%%%%%%%%%%%%%%%%%%%%%%%
%%%%%%%%%%%%%%%%%%%%%%%%%%%%%%%%%%%%%%%%%%
% http://everything2.com/title/Sil+Lum+Tao

Siu-Nim-Tao bedeutet so etwas wie \textit{kleine Ideen Form}. Klein deshalb, da man recht isolierte Bewegungen ausf\"uhrt und sich auf das kleine, die Details konzentriert. So gibt es in dieser Form z.B. keinerlei Wendungen oder Schritte, der Oberk\"orper und die Beine bleiben durchgehend in der selben Position und bewegen sich nicht.

Das Motto der SNT ist \textit{Punkt-Punkt-Klar}. Es sollen jede Stationen (Punkte) der Bewegungen klar ersichtlich erreicht werden. Ein Nehmen von Abk\"urzungen ist also nicht angebracht und verf\"alschen den Ablauf.
 
 %Siu Lim Tao, Sil Lum Tao
 % TODO es lehrt einem wo seine mitte ist, man lernt wo mein koerper aufhoert; guter stand, sich wurzeln (zehen in den boden) siehe IRAS

%CK ... brueckensuchende form

asdf

asdf

asdf

asdf

asdf

\begin{WTSatz}{Schneidende Arme}% 1. SATZ
%%%%%%%%%%%%%%%%%%%%%%%%%%%%%%%%%%%%%%%%%%

	\begin{WTSatzTeil}{Gau-Cha-Tan-Sao}{Gekreuzte Handfl\"ache nach oben Arme}
		\WTGalerySlideshowFour{siunimtao/1-1}{siunimtao/1-2}{siunimtao/1-3}{siunimtao/1-3b}
		
		Zuerst werden die Arme wie in a) zu sehen.
	\end{WTSatzTeil}
	
	\begin{WTSatzTeil}{Gau-Cha-Gan-Sao}{Gekreuzter schneidender Arm}
		\WTGalerySlideshowTwo{siunimtao/1-3}{siunimtao/1-4}
		
		Dann schneiden.
	\end{WTSatzTeil}
	
	\begin{WTSatzTeil}{Kwan-Sao}{?}
		\WTGalerySlideshowThree{siunimtao/1-4}{siunimtao/1-5}{siunimtao/1-6}
		
		Hochdrehen. ACHTUNG NOOB: Gelenke nicht abwinkeln.
	\end{WTSatzTeil}
	
	\begin{WTSatzTeil}{Sau-Kuen}{?}
		\WTGalerySlideshowThree{siunimtao/1-6}{siunimtao/1-7}{siunimtao/1-8}
		
		Zurueckziehen.
	\end{WTSatzTeil}

\end{WTSatz}


\subsubsection{Einfacher Fauststo{\ss}} % 2. SATZ
%%%%%%%%%%%%%%%%%%%%%%%%%%%%%%%%%%%%%%%%%%

Linker und rechter Yat-Gee-Chung-Kuen, Huen-Sao, Sao-Kuen

\subsubsection{Dreifache Verehrung Buddhas} % 3. SATZ
%%%%%%%%%%%%%%%%%%%%%%%%%%%%%%%%%%%%%%%%%%

Tan-Sao, Bon-Huen-Sao, Wu-Sao
Fook-Sao, Bon-Huen-Sao, Wu-Sao x3
Djark-Cheung, Ching-Cheung, Huen-Sao

\subsubsection{Lange Br\"ucke} % 4. SATZ
%%%%%%%%%%%%%%%%%%%%%%%%%%%%%%%%%%%%%%%%%%

Jor-/Yau-Gam-Sao, Chin-/Hau-Gam-Sao
Shang-Lan-Sao, Shang-Fak-Sao
Shang-Lan-Sao, Shang-Djam-Sao
Shang-Tan-Sao, Shang-Djat-Sao, Shang-Biu-Tze-Sao
Cheung-Kiu-Gam-Sao, Shang-Tai-Sao, Sao-Kuen

\subsubsection{Seitlicher Pak-Sao} % 5. SATZ
%%%%%%%%%%%%%%%%%%%%%%%%%%%%%%%%%%%%%%%%%%

Djark-Cheung, Wang-Cheung
Huen-Sao, Sao-Kuen

\subsubsection{Ein D malen} % 6. SATZ
%%%%%%%%%%%%%%%%%%%%%%%%%%%%%%%%%%%%%%%%%%

Tan-Sao, Djam-Sao, Gwat-Sao
Lau-Sao, Ko-Tan-Sao (manche: Tok-Sao), Dai-Cheung
Huen-Sao, Sao-Kuen

\subsubsection{Fl\"ugelarme ausbreiten} % 7. SATZ
%%%%%%%%%%%%%%%%%%%%%%%%%%%%%%%%%%%%%%%%%%

Bong-Sao, Tan-Sao, Ong-Cheung
Tan-Sao, Huen-Sao, Sao-Kuen

\subsubsection{Befreiungssatz} % 8. SATZ
%%%%%%%%%%%%%%%%%%%%%%%%%%%%%%%%%%%%%%%%%%

Tuet-Sao x3
Lin-Wan-Chung-Kuen x3
Huen-Sao, Sao-Sik



%\subsection{Anwendungen zur Siu-Nim-Tao}
%%%%%%%%%%%%%%%%%%%%%%%%%%%%%%%%%%%%%%%%%%
%%%%%%%%%%%%%%%%%%%%%%%%%%%%%%%%%%%%%%%%%%

%Bla bla.
