
\documentclass[twoside]{book}

\title{Das WingTsun Ein-Mal-Eins}



%%
% common packages
%%%%%%%%%%%%%%%%%%%%%%%%%%%%%%%%%%%%

\usepackage[hmarginratio=2:3]{geometry}% odd pages haben rechts einen groesseren abstand, even pages links
\usepackage[austrian]{babel} % setzt zb ToC und listoffigures text
\usepackage[latin1]{inputenc}
\usepackage{hyperref}
\usepackage{url}
\usepackage[font=small,format=plain,labelfont=bf,up,textfont=it,up]{caption}
\usepackage{enumitem}
\usepackage{xcolor}
\usepackage{color}
\usepackage{graphicx}% fuer DeclareGraphicsRule
	%\usepackage[pdftex]{graphicx}
\usepackage{changepage}% to increase side margin; used for cite text
\usepackage{tocloft}% control the design of toc, figures and tables; define own list-of-something
\usepackage{sidecap}% for \begin{SCfigure}
\usepackage{etoolbox}%\usepackage{ifthen}

%%
% misc adjustments
%%%%%%%%%%%%%%%%%%%%%%%%%%%%%%%%%%%%

% change default font family to sans-serif
\renewcommand{\familydefault}{\sfdefault}

%\renewcommand{\contentsname}{asdf does not work to rename toc title} 
% only display chapters/sections that deep;
\setcounter{tocdepth}{2} % TODO set tocdepth to 1 afterwards; or create 2nd (mini)toc with full/less details

% include von graphics definieren
\DeclareGraphicsRule{.tif}{png}{.png}{`convert #1 `dirname #1`/`basename #1 .tif`.png}


% TODO funkt nicht \renewcommand{\figurename}{Bild}

\setlength{\cftfignumwidth}{3em}% hat was zu tun mit tocloft package
%\renewcommand{\cftchapfont}{Steht vorm Kapitel im ToC }



\bibliographystyle{alpha}




%%
% define own list of videos
%%%%%%%%%%%%%%%%%%%%%%%%%%%%%%%%%%%%

\newcommand{\listvideo}{Videoverzeichnis}
\newlistof{video}{exp}{\listvideo}

\newcommand{\WTVideo}[1]{%
\refstepcounter{video}
\addcontentsline{exp}{video}
{\protect\numberline{\thevideo}#1}\par}



%%
% glossar erstellen
%%%%%%%%%%%%%%%%%%%%%%%%%%%%%%%%%%%%

% [toc] [xindy] instead makeindex
% [nonumberlist] notwendig wedgen \glsaddall
\usepackage[nonumberlist]{glossaries} % http://en.wikibooks.org/wiki/LaTeX/Glossary
\makeglossaries

% http://en.wikibooks.org/wiki/LaTeX/Glossary
% http://www.tawingtsun-salzlandkreis.de/begriffserklaerungen.html
% http://www.wingchunkungfu.de/wingchun-lexikon/inhalt.html
% http://www.fit-und-sicher.com/index.php?mainmenu=5&submenu=30&PHPSESSID=oufuvo6ffpf023ki0us32hm9i6


%\renewcommand*{\glspostdescription}{}%den schirchen punkt hinterm text wegnehmen


	%sort=Komputa


% ================================================================

\newglossaryentry{GLcheung}
{
  name=Cheung,
  description={Handfl\"achenstoss}
}

\newglossaryentry{GLkungfu}
{
  name=Kung Fu,
  description={Harte Arbeit, K\"onnen; nicht Kampfkunst was eigentlich wushu}
}



\newglossaryentry{GLwushu}
{
  name=Wushu,
  description={Kampfkunst, und nicht kungfu}
}


\newacronym{GLAsnt}{SNT}
	{Kurz f\"ur \textbf{S}iu-\textbf{N}im-\textbf{T}ao.}

\newacronym{GLAewto}{EWTO}
	{Kurz f\"ur \textbf{E}urop\"aische \textbf{W}ing\textbf{T}sun \textbf{O}rganisation}

\newacronym{GLAiwta}{IWTA}
	{Kurz f\"ur \textbf{I}nternational \textbf{W}ing\textbf{T}sun \textbf{A}ssociation}



% FAMILIE, ANREDEN
% ================================================================

\newglossaryentry{GLsifu}
{
  name=Sifu,
  description={Die Anrede f\"ur den v\"aterlichen Lehrer.},
  plural=Sifus
}
\newglossaryentry{GLsihing}
{
  name=Sihing,
  description={Die Anrede f\"ur den \"alteren Bruder. \"Alter aber im Sinne von l\"anger als man selbst Mitglied der Schule, und nicht notwendigerweise in Menschenjahre \"alter oder Fertigkeitenm\"a{\ss}ig \"uberlegen.}
}

% TECHNIKEN, POSITIONEN
% ================================================================

\newglossaryentry{GLtansao}
{
  name=Tan Sao,
  description={Ist eine Position mit der Handfl\"ache nach oben.}
	%plural=Linuces
	%sort=Komputa
}
\newglossaryentry{GLbongsao}
{
  name=Bong Sao,
  description={Ist eine Position die mit Fl\"ugel- oder Schwingenarm \"ubersetzt wird.}
}

\newacronym{GLAiras}{IRAS}
	{Nach \textbf{I}nnen \textbf{R}otierte \textbf{A}dduktoren \textbf{S}tand.}

%\usepackage{makeglos}
%\usepackage{glossary}
%\makeglossary


%%

% index erstellen
%%%%%%%%%%%%%%%%%%%%%%%%%%%%%%%%%%%%

\usepackage{imakeidx}
\makeindex

%\usepackage{imakeidx}
%\makeindex[columns=2]
%\makeindex[title=My custom index title,columns=2]

%\usepackage{makeidx}
%\usepackage[columns=2,totoc=false]{idxlayout}
%\makeindex



%%
% includes
%%%%%%%%%%%%%%%%%%%%%%%%%%%%%%%%%%%%


%\definecolor{grey}{rgb}{0.9,0.9,0.9} 

\def\WTCommonCite #1#2{
	\begin{adjustwidth}{1cm}{1cm}
	\textit{``#2''}
	\flushright{--- #1}
\end{adjustwidth}
}

	
\newenvironment{WTCommonNoob}
	{
		\definecolor{WTXCommonColoredBoxVariable}{rgb}{1.0,0.3,0.3}
		\begin{WTXCommonColorBoxEnv}{common_noob_alert_sign}{Darauf ist zu achten!}
	} {
		\end{WTXCommonColorBoxEnv}
	}


\newenvironment{WTCommonBegriff}
	{
		\definecolor{WTXCommonColoredBoxVariable}{rgb}{0.96,0.97,0.98}
		\begin{WTXCommonColorBoxEnv}{common_begriff_sign}{Technicus Terminus!}
	} {
		\end{WTXCommonColorBoxEnv}
	}

\newenvironment{WTCommonPruefen}
	{
		\definecolor{WTXCommonColoredBoxVariable}{rgb}{0.80,0.91,0.98} % TODO gscheite farbe noch waehlen fuer Pruefen (irgendwas gelbes, als achtung maessig)
		\begin{WTXCommonColorBoxEnv}{common_begriff_sign}{Korrektheit \"Uberpr\"ufen!}
	} {
		\end{WTXCommonColorBoxEnv}
	}


\definecolor{WTXCommonColoredBoxVariable}{rgb}{1.0,0.0,0.0} % start with default color REEEED


% =================== INTERNAL

\makeatletter\newenvironment{WTXCommonColoredBox}
	{
		\begin{lrbox}{\@tempboxa}\begin{minipage}[b]{400px}
	} {
   		\end{minipage}\end{lrbox}%\columnwidth statt 400px
   		\colorbox{WTXCommonColoredBoxVariable}{\usebox{\@tempboxa}}
	}\makeatother

%\makeatletter\newenvironment{WTXCommonColoredBox}[1][]{%
%   \begin{lrbox}{\@tempboxa}\begin{minipage}{400px}}{\end{minipage}\end{lrbox}%\columnwidth statt 400px
%   \colorbox[HTML]{FF00FF}{\usebox{\@tempboxa}}
%}\makeatother


\newenvironment{WTXCommonColorBoxEnv}[2]
	{
		\begin{flushleft}
			\begin{tabular}{lc}
				\includegraphics[width=32px]{resources/images/icons/#1}
				&
				\begin{WTXCommonColoredBox}
				\textbf{#2} \vspace{3px} \\
	} {
				\end{WTXCommonColoredBox}
			\end{tabular}
		\end{flushleft}
	}


%\colorbox{red}{Black text on red background}
\def\WTXCommonColorBox #1#2#3#4{
	\begin{flushleft}
		\begin{tabular}{lc}
			\includegraphics[width=32px]{resources/images/icons/#2}
			&
			\colorbox[HTML]{#1} {
				\begin{minipage}[b]{400px}
					\textbf{#3}\\ #4
				\end{minipage}
			}
		\end{tabular}
	\end{flushleft} 
}


%\colorbox{grey}{
%	\parbox[t]{1.0\linewidth}{
%		\printtitle 
%		\vspace*{0.7cm}
%	}
%}

% \WTGaleryResetSlideshowCounter
% \WTGalerySlideshowTwo{img1}{img2}
% \WTGalerySlideshowThree{img1}{img2}{img3}
% \WTGalerySlideshowFour{img1}{img2}{img3}{img4}
% \WTGaleryImage{img}{6cm}{my caption}
% \WTGaleryImageTwo{img1}{img2}{my caption}


\def\WTGaleryResetSlideshowCounter{
	\setcounter{WTXGalery_CNT_images}{1}
}

\def \WTGalerySlideshowTwo #1#2{
	\begin{WTXGaleryImagesEnv}{cc}
			\WTXGaleryIncludeImage{#1} & \WTXGaleryIncludeImage{#2} \\
			\WTXGaleryUseAndIncrementCounter{WTXGalery_CNT_images} & \WTXGaleryUseAndIncrementCounter{WTXGalery_CNT_images}  \\
	\end{WTXGaleryImagesEnv}
}
\def \WTGalerySlideshowThree #1#2#3{
	\begin{WTXGaleryImagesEnv}{ccc}
		\WTXGaleryIncludeImage{#1} & \WTXGaleryIncludeImage{#2} & \WTXGaleryIncludeImage{#3} \\
		\WTXGaleryUseAndIncrementCounter{WTXGalery_CNT_images} & \WTXGaleryUseAndIncrementCounter{WTXGalery_CNT_images} & \WTXGaleryUseAndIncrementCounter{WTXGalery_CNT_images} \\
	\end{WTXGaleryImagesEnv}
}
\def \WTGalerySlideshowFour #1#2#3#4{
	\begin{WTXGaleryImagesEnv}{cccc}
		\WTXGaleryIncludeImage{#1} & \WTXGaleryIncludeImage{#2} & \WTXGaleryIncludeImage{#3} & \WTXGaleryIncludeImage{#4} \\
		\WTXGaleryUseAndIncrementCounter{WTXGalery_CNT_images} & \WTXGaleryUseAndIncrementCounter{WTXGalery_CNT_images} & \WTXGaleryUseAndIncrementCounter{WTXGalery_CNT_images} & \WTXGaleryUseAndIncrementCounter{WTXGalery_CNT_images} \\
	\end{WTXGaleryImagesEnv}
}


\def\WTGaleryImage #1#2#3{
	\begin{figure}[htbp]
		\centering
		\includegraphics[width=#2]{resources/images/#1}
		\caption{#3}
	\end{figure}
}
\def\WTGaleryImageTwo #1#2#3{
	\begin{figure}[htbp]
		\centering
		\includegraphics[width=7cm]{resources/images/#1}
		\includegraphics[width=7cm]{resources/images/#2}
		\caption{#3}
	\end{figure}
}

% INTERNAL

\newcounter{WTXGalery_CNT_images}

\newenvironment{WTXGaleryImagesEnv}[1]
	{\begin{figure}[!h]\centering\begin{tabular}{#1}}
	{\end{tabular}\end{figure}}
	
\def\WTXGaleryIncludeImage #1{
	\includegraphics[width=3.3cm]{resources/images/#1}
}
\def\WTXGaleryUseAndIncrementCounter #1{
	\alph{#1})
	\stepcounter{#1}
}

\usepackage{media9}


% sample usage:
% \WTMediaInclude{\WTMediaRefSNT}{my caption}{1:30}

\def\WTMediaInclude#1#2#3{
	\begin{WTXMediaImageOrVideo}{\WTXMediaVariantCaption{#2}{#3}}
		\WTXMediaVariantInclude{#1}
%		\caption{\WTXMediaVariantCaption{#2}{#3}}
	\end{WTXMediaImageOrVideo}
}


%% Concoct some colours of our own
\definecolor[named]{WTColorHeaderLineColor}{HTML}{FF0000}
\definecolor[named]{WTColorHeaderBackground}{HTML}{FFD5D2}
\definecolor[named]{WTColorFooterBackground}{HTML}{FF0000}
\definecolor[named]{WTColorChapterHeaderLightBackground}{HTML}{FFFFFF}

\usepackage{wallpaper}
\usepackage{changepage}
\usepackage[explicit]{titlesec}
\usepackage{fancyhdr}
\usepackage{tikz}
\usetikzlibrary{shapes,positioning}

\usepackage{geometry}
\geometry{
   paperwidth=216mm, paperheight=303mm,
   left=23mm,  %% or inner=23mm
   right=18mm, %% or outer=18mm
   top=23mm, bottom=23mm,
   headheight=\baselineskip,
   headsep=7mm,
   footskip=10mm
}


%% Command to hold chapter illustration image
\newcommand\chapterillustration{}

%% Define how the chapter title is printed
\titleformat{\chapter}{}{}{0pt}{
%% Background image at top of page
\ThisULCornerWallPaper{1}{resources/images/chapters/\chapterillustration}
%% Draw a semi-transparent rectangle across the top
\tikz[overlay,remember picture]
  \fill[WTColorChapterHeaderLightBackground,opacity=.7]
  (current page.north west) rectangle 
  ([yshift=-3cm] current page.north east);
  %% Check if on an odd or even page
  \strictpagecheck\checkoddpage
  %% On odd pages, "logo" image at lower right
  %% corner; Chapter number printed near spine
  %% edge (near the left); chapter title printed
  %% near outer edge (near the right).
  \ifoddpage{
    \ThisLRCornerWallPaper{.35}{resources/images/common/chapter_underlay}
    \begin{tikzpicture}[overlay,remember picture]
    \node[anchor=south west,
      xshift=20mm,yshift=-30mm,
      font=\sffamily\bfseries\huge] 
      at (current page.north west) 
      {\chaptername\ \thechapter};
    \node[fill=WTColorFooterBackground!80!black,text=white,
      font=\Huge\bfseries, 
      inner ysep=12pt, inner xsep=20pt,
      rounded rectangle,anchor=east, 
      xshift=-20mm,yshift=-30mm] 
      at (current page.north east) {#1}; % #1 ... chapter titel
    \end{tikzpicture}
  }
  %% On even pages, "logo" image at lower left
  %% corner; Chapter number printed near outer
  %% edge (near the right); chapter title printed
  %% near spine edge (near the left).
  \else {
    \ThisLLCornerWallPaper{.35}{resources/images/common/chapter_underlay}
    \begin{tikzpicture}[overlay,remember picture]
    \node[anchor=south east,
      xshift=-20mm,yshift=-30mm,
      font=\sffamily\bfseries\huge] 
      at (current page.north east)
      {\chaptername\ \thechapter};
    \node[fill=WTColorFooterBackground!80!black,text=white,
      font=\Huge\bfseries,
      inner sep=12pt, inner xsep=20pt,
      rounded rectangle,anchor=west,
      xshift=20mm,yshift=-30mm] 
      at ( current page.north west) {#1}; % #1 ... chapter titel
    \end{tikzpicture}
  }
  \fi
}
\titlespacing*{\chapter}{0pt}{0pt}{135mm}

%% Set the uniform width of the colour box
%% displaying the page number in footer
%% to the width of "99"
\newlength\pagenumwidth
\settowidth{\pagenumwidth}{99}

%% Define style of page number colour box
\tikzset{pagefooter/.style={
anchor=base,font=\sffamily\bfseries\small,
text=white,fill=WTColorFooterBackground!80!black,text centered,
text depth=17mm,text width=\pagenumwidth}}


%%%%%%%%%%
%%% Re-define running headers on non-chapter pages
%%%%%%%%%%


\fancypagestyle{headings}{%
  \fancyhf{}   % Clear all headers and footers first
  %% Right headers on odd pages
  \fancyhead[RO]{%
    %% First draw the background rectangles
    \begin{tikzpicture}[remember picture,overlay]
    \fill[WTColorHeaderBackground!25!white] (current page.north east) rectangle (current page.south west);
    \fill[white, rounded corners] ([xshift=-10mm,yshift=-20mm]current page.north east) rectangle ([xshift=15mm,yshift=17mm]current page.south west);
    \end{tikzpicture}
    %% Then the decorative line and the right mark
    \begin{tikzpicture}[xshift=-.75\baselineskip,yshift=.25\baselineskip,remember picture,    overlay,fill=WTColorHeaderLineColor,draw=WTColorHeaderLineColor]\fill circle(3pt);\draw[semithick](0,0) -- (current page.west |- 0,0);\end{tikzpicture} \sffamily\itshape\small\nouppercase{\rightmark}%\rightmark; \thesection=1.2  \sectionname
  }

  %% Left headers on even pages
  \fancyhead[LE]{%
    %% Background rectangles first
    \begin{tikzpicture}[remember picture,overlay]
    \fill[WTColorHeaderBackground!25!white] (current page.north east) rectangle (current page.south west);
    \fill[white, rounded corners] ([xshift=-15mm,yshift=-20mm]current page.north east) rectangle ([xshift=10mm,yshift=17mm]current page.south west);
    \end{tikzpicture}
    %% Then the right mark and the decorative line
    \sffamily\itshape\small\nouppercase{\leftmark}\ 
    \begin{tikzpicture}[xshift=.5\baselineskip,yshift=.25\baselineskip,remember picture, overlay,fill=WTColorHeaderLineColor,draw=WTColorHeaderLineColor]\fill (0,0) circle (3pt); \draw[semithick](0,0) -- (current page.east |- 0,0 );\end{tikzpicture}
  }

  %% Right footers on odd pages and left footers on even pages,
  %% display the page number in a colour box
  \fancyfoot[RO,LE]{\tikz[baseline]\node[pagefooter]{\thepage};}
  \renewcommand{\headrulewidth}{0pt}
  \renewcommand{\footrulewidth}{0pt}
}



\fancypagestyle{preheadings}{%
  \fancyhf{}   % Clear all headers and footers first
  %% Right headers on odd pages
  \fancyhead[RO]{%
    %% First draw the background rectangles
    \begin{tikzpicture}[remember picture,overlay]
    \fill[WTColorHeaderBackground!25!white] (current page.north east) rectangle (current page.south west);
    \fill[white, rounded corners] ([xshift=-10mm,yshift=-20mm]current page.north east) rectangle ([xshift=15mm,yshift=17mm]current page.south west);
    \end{tikzpicture}
    %% Then the decorative line and the right mark
    \begin{tikzpicture}[xshift=-.75\baselineskip,yshift=.25\baselineskip,remember picture,    overlay,fill=WTColorHeaderLineColor,draw=WTColorHeaderLineColor]\fill circle(3pt);\draw[semithick](0,0) -- (current page.west |- 0,0);\end{tikzpicture} \sffamily\itshape\small\nouppercase{}%\rightmark; \thesection=1.2  \sectionname
  }

  %% Left headers on even pages
  \fancyhead[LE]{%
    %% Background rectangles first
    \begin{tikzpicture}[remember picture,overlay]
    \fill[WTColorHeaderBackground!25!white] (current page.north east) rectangle (current page.south west);
    \fill[white, rounded corners] ([xshift=-15mm,yshift=-20mm]current page.north east) rectangle ([xshift=10mm,yshift=17mm]current page.south west);
    \end{tikzpicture}
    %% Then the right mark and the decorative line
    \sffamily\itshape\small\nouppercase{}\ 
    \begin{tikzpicture}[xshift=.5\baselineskip,yshift=.25\baselineskip,remember picture, overlay,fill=WTColorHeaderLineColor,draw=WTColorHeaderLineColor]\fill (0,0) circle (3pt); \draw[semithick](0,0) -- (current page.east |- 0,0 );\end{tikzpicture}
  }

  %% Right footers on odd pages and left footers on even pages,
  %% display the page number in a colour box
  \fancyfoot[RO,LE]{\tikz[baseline]\node[pagefooter]{\thepage};}
  \renewcommand{\headrulewidth}{0pt}
  \renewcommand{\footrulewidth}{0pt}
}


%%%%%%%%%%
%%% Re-define running headers on chapter pages
%%%%%%%%%%
\fancypagestyle{plain}{%
  %% Clear all headers and footers
  \fancyhf{}
  %% Right footers on odd pages and left footers on even pages,
  %% display the page number in a colour box
  \fancyfoot[RO,LE]{\tikz[baseline]\node[pagefooter]{\thepage};}
  \renewcommand{\headrulewidth}{0pt}
  \renewcommand{\footrulewidth}{0pt}
}

%\renewcommand{\chaptermark}[1]{\markright{#1}{}}
%\renewcommand{\sectionmark}[1]{\markright{#1}{}}

%% Public domain image from
%% http://www.public-domain-image.com/objects/computer-chips/slides/six-computers-chips-circuits.html
%\renewcommand\chapterillustration{six-computers-chips-circuits}







