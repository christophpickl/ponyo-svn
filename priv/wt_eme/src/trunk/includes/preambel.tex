
\documentclass[twoside]{book}

\title{Das WingTsun Ein-Mal-Eins}



%%
% common packages
%%%%%%%%%%%%%%%%%%%%%%%%%%%%%%%%%%%%

\usepackage[hmarginratio=2:3]{geometry}% odd pages haben rechts einen groesseren abstand, even pages links
\usepackage[austrian]{babel} % setzt zb ToC und listoffigures text
\usepackage[latin1]{inputenc}
\usepackage{hyperref}
\usepackage{url}
\usepackage[font=small,format=plain,labelfont=bf,up,textfont=it,up]{caption}
\usepackage{enumitem}
\usepackage{xcolor}
\usepackage{color}
\usepackage{graphicx}% fuer DeclareGraphicsRule
	%\usepackage[pdftex]{graphicx}
\usepackage{changepage}% to increase side margin; used for cite text
\usepackage{tocloft}% control the design of toc, figures and tables; define own list-of-something
\usepackage{sidecap}% for \begin{SCfigure}
\usepackage{etoolbox}%\usepackage{ifthen}

%%
% misc adjustments
%%%%%%%%%%%%%%%%%%%%%%%%%%%%%%%%%%%%

% change default font family to sans-serif
\renewcommand{\familydefault}{\sfdefault}

%\renewcommand{\contentsname}{asdf does not work to rename toc title} 
% only display chapters/sections that deep;
\setcounter{tocdepth}{2} % TODO set tocdepth to 1 afterwards; or create 2nd (mini)toc with full/less details

% include von graphics definieren
\DeclareGraphicsRule{.tif}{png}{.png}{`convert #1 `dirname #1`/`basename #1 .tif`.png}


% TODO funkt nicht \renewcommand{\figurename}{Bild}

\setlength{\cftfignumwidth}{3em}% hat was zu tun mit tocloft package
%\renewcommand{\cftchapfont}{Steht vorm Kapitel im ToC }



\bibliographystyle{alpha}




%%
% define own list of videos
%%%%%%%%%%%%%%%%%%%%%%%%%%%%%%%%%%%%

\newcommand{\listvideo}{Videoverzeichnis}
\newlistof{video}{exp}{\listvideo}

\newcommand{\WTVideo}[1]{%
\refstepcounter{video}
\addcontentsline{exp}{video}
{\protect\numberline{\thevideo}#1}\par}



%%
% glossar erstellen
%%%%%%%%%%%%%%%%%%%%%%%%%%%%%%%%%%%%

% [toc] [xindy] instead makeindex
% [nonumberlist] notwendig wedgen \glsaddall
\usepackage[nonumberlist]{glossaries} % http://en.wikibooks.org/wiki/LaTeX/Glossary
\makeglossaries

% http://en.wikibooks.org/wiki/LaTeX/Glossary

%\renewcommand*{\glspostdescription}{}%den schirchen punkt hinterm text wegnehmen


\newglossaryentry{GLcomputer}
{
  name=computer,
  description={is a programmable machine that receives input,
               stores and manipulates data, and provides
               output in a useful format}
	%plural=Linuces
	%sort=Komputa
}


\newacronym{GLAiras}{IRAS}
	{Innen Rotierte Adduktoren Stand}

%\usepackage{makeglos}
%\usepackage{glossary}
%\makeglossary


%%

% index erstellen
%%%%%%%%%%%%%%%%%%%%%%%%%%%%%%%%%%%%

\usepackage{imakeidx}
\makeindex

%\usepackage{imakeidx}
%\makeindex[columns=2]
%\makeindex[title=My custom index title,columns=2]

%\usepackage{makeidx}
%\usepackage[columns=2,totoc=false]{idxlayout}
%\makeindex



%%
% includes
%%%%%%%%%%%%%%%%%%%%%%%%%%%%%%%%%%%%


%\definecolor{grey}{rgb}{0.9,0.9,0.9} 


% fuer \WTKurzSatz in formen.tex, und \WTEWTOLehrer in ewto.tex
\def\WTXCommonImageTop#1{\vtop{\null\hbox{#1}}}

%%
% table additions
%%%%%%%%%%%%%%%%%%%%%%%%%%%%%%%%%%%%

\newcommand{\WTXcell}[2]{\begin{tabular}[t]{@{}#2@{}}#1\end{tabular}}
\newcommand{\WTcell}[1]{\WTXcell{#1}{l}}%left
\newcommand{\WTcellCentered}[1]{\WTXcell{#1}{c}}


%%
% define custom enumeration style
%%%%%%%%%%%%%%%%%%%%%%%%%%%%%%%%%%%%

\newlist{WTalphenum}{enumerate}{2}
\setlist[WTalphenum,1]{label=\alph*)}
\setlist[WTalphenum,2]{label=\roman*)}

\newenvironment{WTalphenumNarrow}{
\begin{WTalphenum}
  \setlength{\itemsep}{1pt}
  \setlength{\parskip}{2pt}
  \setlength{\parsep}{0pt}
}{\end{WTalphenum}}
	

%%%%%%%%%%%%%%%%%%%%%%%%%%%%%%%%%%%%


\def \quote #1 {``#1''}
\def \quoteit #1 {``\textit{#1}''}

\newenvironment{enumerateNarrow}{
\begin{enumerate}
  \setlength{\itemsep}{1pt}
  \setlength{\parskip}{2pt}
  \setlength{\parsep}{0pt}
}{\end{enumerate}}

\newenvironment{itemizeNarrow}{
\begin{itemize}
  \setlength{\itemsep}{1pt}
  \setlength{\parskip}{2pt}
  \setlength{\parsep}{0pt}
}{\end{itemize}}


\def\WTCommonCite #1#2{
	\begin{adjustwidth}{1cm}{1cm}
	\textit{``#2''}
	\flushright{--- #1}
\end{adjustwidth}
}

	
\newenvironment{WTCommonNoob}
	{
		\definecolor{WTXCommonColoredBoxVariable}{rgb}{1.0,0.3,0.3}
		\begin{WTXCommonColorBoxEnv}{common_noob_alert_sign}{Darauf ist zu achten!}
	} {
		\end{WTXCommonColorBoxEnv}
	}


\newenvironment{WTCommonBegriff}
	{
		\definecolor{WTXCommonColoredBoxVariable}{rgb}{0.96,0.97,0.98}
		\begin{WTXCommonColorBoxEnv}{common_begriff_sign}{Technicus Terminus!}
	} {
		\end{WTXCommonColorBoxEnv}
	}

%\newenvironment{WTCommonPruefen}
%	{
%		\definecolor{WTXCommonColoredBoxVariable}{rgb}{0.80,0.91,0.98} % TODO gscheite farbe noch waehlen fuer Pruefen (irgendwas gelbes, als achtung maessig)
%		\begin{WTXCommonColorBoxEnv}{common_begriff_sign}{Korrektheit \"Uberpr\"ufen!}
%	} {
%		\end{WTXCommonColorBoxEnv}
%	}


\definecolor{WTXCommonColoredBoxVariable}{rgb}{1.0,0.0,0.0} % start with default color REEEED


% =================== INTERNAL

\makeatletter\newenvironment{WTXCommonColoredBox}
	{
		\begin{lrbox}{\@tempboxa}\begin{minipage}[b]{400px}
	} {
   		\end{minipage}\end{lrbox}%\columnwidth statt 400px
   		\colorbox{WTXCommonColoredBoxVariable}{\usebox{\@tempboxa}}
	}\makeatother

%\makeatletter\newenvironment{WTXCommonColoredBox}[1][]{%
%   \begin{lrbox}{\@tempboxa}\begin{minipage}{400px}}{\end{minipage}\end{lrbox}%\columnwidth statt 400px
%   \colorbox[HTML]{FF00FF}{\usebox{\@tempboxa}}
%}\makeatother


\newenvironment{WTXCommonColorBoxEnv}[2]
	{
		\begin{flushleft}
			\begin{tabular}{lc}
				\includegraphics[width=32px]{resources/images/icons/#1}
				&
				\begin{WTXCommonColoredBox}
				\textbf{#2} \vspace{3px} \\
	} {
				\end{WTXCommonColoredBox}
			\end{tabular}
		\end{flushleft}
	}


%\colorbox{red}{Black text on red background}
\def\WTXCommonColorBox #1#2#3#4{
	\begin{flushleft}
		\begin{tabular}{lc}
			\includegraphics[width=32px]{resources/images/icons/#2}
			&
			\colorbox[HTML]{#1} {
				\begin{minipage}[b]{400px}
					\textbf{#3}\\ #4
				\end{minipage}
			}
		\end{tabular}
	\end{flushleft} 
}


%\colorbox{grey}{
%	\parbox[t]{1.0\linewidth}{
%		\printtitle 
%		\vspace*{0.7cm}
%	}
%}

% \WTGaleryResetSlideshowCounter
% \WTGalerySlideshowTwo{img1}{img2}
% \WTGalerySlideshowThree{img1}{img2}{img3}
% \WTGalerySlideshowFour{img1}{img2}{img3}{img4}
% \WTGaleryImage{img}{6cm}{my caption}
% \WTGaleryImageTwo{img1}{img2}{my caption}


\def\WTGaleryResetSlideshowCounter{
	\setcounter{WTXGalery_CNT_images}{1}
}

\def \WTGalerySlideshowTwo #1#2{
	\begin{WTXGaleryImagesEnv}
			\WTXGaleryIncludeImage{#1} \WTXGaleryIncludeImage{#2}
	\end{WTXGaleryImagesEnv}
}
\def \WTGalerySlideshowThree #1#2#3{
	\begin{WTXGaleryImagesEnv}
		\WTXGaleryIncludeImage{#1} \WTXGaleryIncludeImage{#2} \WTXGaleryIncludeImage{#3}
	\end{WTXGaleryImagesEnv}
}
\def \WTGalerySlideshowFour #1#2#3#4{
	\begin{WTXGaleryImagesEnv}
		\WTXGaleryIncludeImage{#1} \WTXGaleryIncludeImage{#2} \WTXGaleryIncludeImage{#3} \WTXGaleryIncludeImage{#4}
	\end{WTXGaleryImagesEnv}
}


\def\WTGaleryImage #1#2#3{
	\begin{figure}[htbp]
		\centering
		\includegraphics[width=#2]{resources/images/#1}
		\caption{#3}
	\end{figure}
}
\def\WTGaleryImageTwo #1#2#3{
	\begin{figure}[htbp]
		\centering
		\includegraphics[width=7cm]{resources/images/#1}
		\includegraphics[width=7cm]{resources/images/#2}
		\caption{#3}
	\end{figure}
}

% INTERNAL

\newcounter{WTXGalery_CNT_images}

\newenvironment{WTXGaleryImagesEnv}
	{\begin{center}}
	{\end{center}}
	
\def\WTXGaleryIncludeImage #1{
	\begin{picture}(110,80)
		\put(0,0){\includegraphics[width=110px]{resources/images/#1}}
		\put(0,3){\colorbox{black}{\color{black}{ \rule{4pt}{9pt} }}}
		\put(4,4){\hbox{\textbf{\textcolor{red}{\WTXGaleryUseAndIncrementCounter}}}}
%		\put(0,3){\framebox{\phantom{\rule{40pt}{40pt}}}} %Dummy box in place of an image.
	\end{picture}
}

\def\WTXGaleryUseAndIncrementCounter {
	\small{\alph{WTXGalery_CNT_images})}
%	\arabic{#1})
	\stepcounter{WTXGalery_CNT_images}
}

\usepackage{media9}


% sample usage:
% \WTMediaInclude{\WTMediaRefSNT}{my caption}{1:30}

\def\WTMediaInclude#1#2#3{
	\begin{WTXMediaImageOrVideo}{\WTXMediaVariantCaption{#2}{#3}}
		\WTXMediaVariantInclude{#1}
%		\caption{\WTXMediaVariantCaption{#2}{#3}}
	\end{WTXMediaImageOrVideo}
}


%\usepackage{fancyhdr}\pagestyle{fancy}
%\fancyhf{}
%\renewcommand{\headrulewidth}{0pt}
%\renewcommand{\footrulewidth}{1pt}
%
%\lhead{\leftmark}
%\chead{}
%\rhead{\small TU Wien, Theoretische Informatik 2}
%
%\fancyfoot[L]{}
%\fancyfoot[C]{\small \thepage/\pageref{LastPage}}
%\fancyfoot[R]{} 

\usepackage{fancyhdr}\pagestyle{fancy}
\usepackage{lastpage}
%\setlength{\headheight}{15pt}
% 
%\renewcommand{\chaptermark}[1]{\markboth{#1}{}}
%\renewcommand{\sectionmark}[1]{\markright{#1}{}}



\fancyhf{}
\renewcommand{\headrulewidth}{0pt}
\renewcommand{\footrulewidth}{0pt}

\lhead{}%\leftmark}
\chead{}
\rhead{}%\rightmark}%\small WT EME

\fancyfoot[C]{}
\fancyfoot[RO,LE]{\small \thepage/\pageref{LastPage}}
\fancyfoot[RE,LO]{}



