
% http://en.wikibooks.org/wiki/LaTeX/Glossary
% http://www.tawingtsun-salzlandkreis.de/begriffserklaerungen.html
% http://www.wingchunkungfu.de/wingchun-lexikon/inhalt.html

%\renewcommand*{\glspostdescription}{}%den schirchen punkt hinterm text wegnehmen


	%sort=Komputa


% ================================================================

\newglossaryentry{GLcheung}
{
  name=Cheung,
  description={Handfl\"achenstoss}
}

\newglossaryentry{GLkungfu}
{
  name=Kung Fu,
  description={Harte Arbeit, K\"onnen; nicht Kampfkunst was eigentlich wushu}
}

\newglossaryentry{GLwushu}
{
  name=Wushu,
  description={Kampfkunst, und nicht kungfu}
}



% FAMILIE, ANREDEN
% ================================================================

\newglossaryentry{GLsifu}
{
  name=Sifu,
  description={Die Anrede f\"ur den v\"aterlichen Lehrer.},
  plural=Sifus
}
\newglossaryentry{GLsihing}
{
  name=Sihing,
  description={Die Anrede f\"ur den \"alteren Bruder. \"Alter aber im Sinne von l\"anger als man selbst Mitglied der Schule, und nicht notwendigerweise in Menschenjahre \"alter oder Fertigkeitenm\"a{\ss}ig \"uberlegen.}
}

% TECHNIKEN, POSITIONEN
% ================================================================

\newglossaryentry{GLtansao}
{
  name=Tan Sao,
  description={Ist eine Position mit der Handfl\"ache nach oben.}
	%plural=Linuces
	%sort=Komputa
}
\newglossaryentry{GLbongsao}
{
  name=Bong Sao,
  description={Ist eine Position die mit Fl\"ugel- oder Schwingenarm \"ubersetzt wird.}
}

\newacronym{GLAiras}{IRAS}
	{Nach \textbf{I}nnen \textbf{R}otierte \textbf{A}dduktoren \textbf{S}tand.}