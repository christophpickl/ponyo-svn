

\newenvironment{WTSatz}[1]
	{\WTGaleryResetSlideshowCounter \subsection{#1}}
	{}

\newenvironment{WTSatzTeil}[2]
	{\paragraph{#1} (\textit{#2})}
	{}

\def\WTXFormen_EingangsGraphics#1{\includegraphics[width=2.5cm]{resources/images/eingangsform/#1}}


\renewcommand\chapterillustration{pushing_minimalistisch}
\chapter{Die Form}

Die Formen setzen sich aus einer Reihe von vordefinierten Bewegungsabl\"aufen zusammen, die von jedem WT-Praktizierenden regelm\"a{\ss}ig, aufmerksam und bewusst durchgef\"uhrt werden sollte. Die Art der Ausf\"uhrung hierbei kann gerne variieren: mal schnell/langsam, mal hart/weich, mal auf dem einen/anderen Bein, mit nur einer Hand, etc. Auf jeden Fall sollte nicht durch Hetzerei die Bewegung \textit{verwischt} werden, sondern jeder Punkt noch durchlaufen werden.

Nach l\"angerem \"Uben schleifen sich die Bewegungen ein und sind sofort f\"ur einen abrufbar. Dies gibt einem die M\"oglichkeit sich eine Art Werkzeugkasten zu bauen, aus dem man sich dann bei Bedarf bedient um eine konkrete Technik umzusetzen.

\newpage


\section{Begin und Schluss}
%%%%%%%%%%%%%%%%%%%%%%%%%%%%%%%%%%%%%%%%%%
%%%%%%%%%%%%%%%%%%%%%%%%%%%%%%%%%%%%%%%%%%




%\subsection{Eingangsform}

Die \textbf{Eingangsform} geht von einem lockeren Stand aus, und f\"uhrt in den sogenannten IRAS\index{IRAS}-Stand, was soviel bedeutet wie: \textit{\textbf{I}nternally \textbf{R}otated \textbf{A}dduction \textbf{S}tance} (oder auch auf Chinesisch \textit{Yee-Gee-Kim-Yeung-Ma} mit w\"ortlicher \"Ubersetzung Schriftzeichen-Zwei-Adduktions-Stand). Hierbei werden die Knie zueinandergepresst wobei die Oberschenkelmuskulator (die Adduktoren; v. lat.: \textit{adducere}, hinf\"uhren, hinziehen) durchgehend gespannt und somit trainiert werden. Dieser Stand versteht sich nur als Trainingsstand zur Verbesserung der Beinmuskulator und ist nicht gedacht zum Einsatz im Ernstfall.

\begin{figure}[htbp]
	\centering
	\begin{tabular}{ccccc}
		\WTXFormen_EingangsGraphics{arm1} & \WTXFormen_EingangsGraphics{arm2} & \WTXFormen_EingangsGraphics{arm3} & \WTXFormen_EingangsGraphics{arm3} & \WTXFormen_EingangsGraphics{arm3} \\
		\WTXFormen_EingangsGraphics{bein1} & \WTXFormen_EingangsGraphics{bein2} & \WTXFormen_EingangsGraphics{bein3} & \WTXFormen_EingangsGraphics{bein4} & \WTXFormen_EingangsGraphics{bein5} \\
		a) & b) & c) & d) & e) \\
	\end{tabular}
\end{figure}

Ausgangspunkt ist ein aufrecher, entspannter Stand a), der Blick ist nach vorne gerichtet, der Kopf sitzt weiter hinten, die Arme h\"angen locker auf der Seite hinunter, die F\"usse sind gschlossen und die H\"ande offen. Nun werden beide H\"ande zu F\"austen gemacht und dann der Seite entlang brusthoch angezogen b), die Fingerkn\"ocheln schauen nicht vor dem K\"orper und die Ellbogen nicht aus ihm heraus, die Unterarme sind parallel zum Boden. Dann werden die Knie gebeugt c) und man setzt sich leicht mit aufrechtem Oberk\"orper in die Hocke. Um den Stand zu verbreitern dreht man die Zehenspitzen so weit wie m\"oglich auf den Fersen auseinander d). Danach folgen die Fersen e) und werden \"uber die Fussmitte aufgemacht sodass beide F\"u{\ss}e ein gleichseitiges Dreieck am Boden formen.

\WTCommonNoob{Man dreht gerne auf den Zehen auseinander was zu einem falschen Stand f\"uhrt.}

\WTGaleryImageTwo{eingangsform/fertig-front}{eingangsform/fertig-seite}{Demonstration des IRAS Stand}

%\subsection{Ausgangsform}

Am Ende gibt es dann eine kurzgehaltene \textbf{Ausgangsform} die wieder zur\"uck in einen lockeren Stand f\"uhrt:

\begin{WTalphenum}
	\item Ausgangspunkt ist der volle IRAS Stand.
	\item Als erstes wird der rechte Fuss gerade gestellt, sodass Zehen und Ferse eine gerade Linie nach vorne bilden.
	\item Nun stellt (noch immer gehockt) der linke Fuss zum rechten parallel Fuss mit kleinem Abstand zu.
	\item Erst jetzt werden die Beine aus der Hocke gestreckt.
	\item Und zu letzt werden die Arme an der Seite locker hinuntergef\"uhrt.
\end{WTalphenum}


\section{Siu-Nim-Tau}
%%%%%%%%%%%%%%%%%%%%%%%%%%%%%%%%%%%%%%%%%%
%%%%%%%%%%%%%%%%%%%%%%%%%%%%%%%%%%%%%%%%%%
% http://everything2.com/title/Sil+Lum+Tao

Siu-Nim-Tau bedeutet so etwas wie \textit{Kleine Idee} oder auch \textit{kleine Ideen Form}. Klein deshalb, da man recht isolierte Bewegungen ausf\"uhrt und sich auf das kleine, die Details konzentriert. So gibt es in dieser Form z.B. keinerlei Wendungen oder Schritte, der Oberk\"orper und die Beine bleiben durchgehend in der selben Position und bewegen sich nicht.

% TODO: ZIEL SNT: entspannt (schulter unten), koerpermitte finden (2. satz zur mitte, zurueckziehen von paks zur mitte), koerperende kennen (seitliche paks, oben durch lau-sau satz 4, unten durch gum-sau satz 4; berechnung der zentrallinie moeglich)

Das Motto der SNT ist \textit{Punkt-Punkt-Klar}. Es sollen jede Stationen (Punkte) der Bewegungen klar ersichtlich erreicht werden. Ein Nehmen von Abk\"urzungen ist also nicht angebracht und verf\"alschen den Ablauf. Weitere Motto die beim \"Uben helfen:

\begin{itemize}
	\item Dr\"ucke deinen Kopf Richtung Himmel und stehe fest am Boden.
	\item Kopf hoch mit horizontalem Blick.
	\item Aufnahmef\"ahiger Brustkorb und aufgerichteter R\"ucken.
	\item H\"ufte gerade und Bauch einziehen
	\item Bei allen Bewegungen der Arme zu beachten: Tiefer Ellbogen und entspannte Schulterhaltung
	\item Wenn eine Armbewegung erfolgt: Schau in die Richtung der Handbewegung.
\end{itemize}

\WTCommonBegriff{Siu-Nim-Tau kann auch geschrieben werden als \textit{Siu-Lin-Tau} (ein bisschen am Anfang \"uben) oder als \textit{Siu-Lam-Tau} (Vergiss nicht, dass der Ursprung des wing chun im Shaolin liegt)}
 

 % TODO es lehrt einem wo seine mitte ist, man lernt wo mein koerper aufhoert; guter stand, sich wurzeln (zehen in den boden) siehe IRAS


\begin{WTSatz}{Schneidende Arme}% 1. SATZ
%%%%%%%%%%%%%%%%%%%%%%%%%%%%%%%%%%%%%%%%%%

\textbf{Ge\"ubte Techniken}: Tan-Sao, Gan-Sao.

	\begin{WTSatzTeil}{Gau-Cha-Tan-Sao}{Gekreuzte Handfl\"achen nach oben H\"ande}
		\WTGalerySlideshowFour{siunimtau/1/1}{siunimtau/1/2}{siunimtau/1/3}{siunimtau/1/3b}
		
		Als allererstes werden die geschlossenen F\"auste a) ganz ge\"offnet sodass die Handfl\"achen eine gerade Oberfl\"ache bilden b). Danach werden die H\"ande gef\"uhrt von den Ellbogen (welche sowieso schon eine gewisse Vorspannung besitzen) nach vorne geschoben c) bis dass ein Abstand von einer Faust breit zwischen K\"orper und Ellbogen besteht, und die Fingerspitzen auf H\"ohe der Schultern sind d). Um korrekte Winkeln zu gew\"ahrleisten sollte noch gepr\"uft werden, ob die Hautlinien beim rechten Handgelenk gerade noch sichtbar sind, und die Ellbogen seitlich mit dem K\"orper abschliessen.
		
		Der sogenannte \textit{gekreuzte Tan-Sao} ist in beidh\"andiger Ausf\"uhrung nur in der Form zu sehen, da er wenn dann nur einarmig und in gewendeter Position Anwendung findet.
		%TODO bild von hautlinie die gerade mit hand abschliesst fuern gekreuzten-tan sao
	\end{WTSatzTeil}
	
	\begin{WTSatzTeil}{Gau-Cha-Gan-Sao}{Gekreuzte schneidende Arme}
		\WTGalerySlideshowTwo{siunimtau/1/3}{siunimtau/1/4}
		
		Startend im doppelten Tan-Sao e) schneidet der Gan-Sao (nicht zu verwechseln mit dem Gam-Sao) mit der Elle eine gerade Linie hinunter f), so als w\"urden die Arme etwas abschneiden wollen, und beschreibt \textbf{nicht} in einer Wischbewegung einen Kreis! Darum wird auch geraten die Bewegung nicht bewusst zu steuern, sondern einfach nur die Arme locker fallen lassen. Am Ende wird der Unterarm leicht nach au{\ss}en routiert so dass man auch wirklich mit dem Knochen auftrifft.
	\end{WTSatzTeil}
	
	\begin{WTSatzTeil}{Kwan-Sao}{Rotierender Arm}
		\WTGalerySlideshowThree{siunimtau/1/4}{siunimtau/1/5}{siunimtau/1/6}
		
		Ausgehend vom fertigen Gan-Sao g) werden die Arme nach oben rotiert h), wobei die H\"ande nahe am K\"orper entlang angehoben werden. Der linke Arm dreht sich dabei unter dem Rechten durch und befindet sich dann weiter innen.
		
		Am Schluss der Bewegung befindet man sich exakt wieder in der urspr\"unglichen Tan-Sao Position i) mit der linken Hand oben.
		
		\WTCommonNoob{Beim Kwan-Sao werden Handgelenke nicht abgewinkelt, sondern bleiben eine geradlinige Verl\"angerung des Unterarms. Um Platz zu machen f\"uhr die L\"ange der Arme muss man die \textbf{Ellb\"ogen} leicht \textbf{nach aussen} schieben.}
	\end{WTSatzTeil}
	
	\begin{WTSatzTeil}{Sau-Kuen}{Ellbogen nach Hinten}
		\WTGalerySlideshowThree{siunimtau/1/6}{siunimtau/1/7}{siunimtau/1/8}
		
		Ausgehend vom Tan-Sao j) wird mit beiden H\"anden gleichzeitig eine Faust gemacht k) und dann stossen die Ellb\"ogen dynamisch nach hinten, sodass die F\"auste wieder auf Brusth\"ohe sind l).
	\end{WTSatzTeil}
	

\end{WTSatz}

%%%%%%%%%%%%%%%%%%%%%%%%%%%%%%%%%%%%%%%%%%

\WTGalerySlideshowThree{siunimtau/2/1}{siunimtau/2/2}{siunimtau/2/3}

\begin{WTSatz}{Einfacher Fauststo{\ss}}% 2. SATZ
	\begin{WTSatzTeil}{Yat-Gee-Chung-Kuen}{Fauststoss}
		\WTGalerySlideshowFour{siunimtau/2/1}{siunimtau/2/2}{siunimtau/2/3}{siunimtau/2/4}
		
		Eindrehen, Zur Mitte, Fauststoss.
		
		Linker und rechter Fauststoss.
		
		Chinesisches Sonnenzeichen, Sonnenfaust, vertikaler Fauststoss.
	\end{WTSatzTeil}
	\begin{WTSatzTeil}{Tan-Sao, Huen-Sao, Sao-Kuen}{Bilden zusammen die Schlusssequenz}
		\WTGalerySlideshowThree{siunimtau/2/5}{siunimtau/2/6}{siunimtau/2/7}
		
		\WTGalerySlideshowThree{siunimtau/2/8}{siunimtau/2/9}{siunimtau/2/10}
		
		Diese \textbf{Schlusssequenz} findet sich am Ende von (fast) jedem Satz und besteht aus den drei Techniken Tan-Sao (handfl\"ache nach Oben Hand), Huen-Sao (zirkelnde Hand) und Sao-Kuen (das Zur\"uckziehen der Ellbogen).
		
		TODO Die Drehung der Faust am Schluss h) - i) nicht immer zwingend erforderlich als Punkt klar ausweisen, manchmal auch waehrend dem zurueckziehen gedreht.
	\end{WTSatzTeil}
\end{WTSatz}

%%%%%%%%%%%%%%%%%%%%%%%%%%%%%%%%%%%%%%%%%%

\begin{WTSatz}{Dreifache Verehrung Buddhas}% 3. SATZ
	
	\WTCommonBegriff{Der dritte Satz wird im Originalen mit \textit{Sam Bai Fat}\index{Sam Bai Fat} \"ubersetzt; oder auch \textit{Fat Shan} in Kantonesisch}
	
	\begin{WTSatzTeil}{Tan-Sao}{Handfl\"ache oben Hand}
		\WTGalerySlideshowFour{siunimtau/3/1}{siunimtau/3/2}{siunimtau/3/3}{siunimtau/3/4}
		
		Asdf.
	\end{WTSatzTeil}
	\begin{WTSatzTeil}{Bon-Huen-Sao}{Halbe zirkelnde Hand}
		\WTGalerySlideshowFour{siunimtau/3/5}{siunimtau/3/6}{siunimtau/3/7}{siunimtau/3/8}
		
		Bon = Halber.
	\end{WTSatzTeil}
	\begin{WTSatzTeil}{Wu-Sao}{Sch\"utzende Hand}
		\WTGalerySlideshowTwo{siunimtau/3/8}{siunimtau/3/9}
		
		asdfr.
	\end{WTSatzTeil}
	\begin{WTSatzTeil}{Fook-Sao, Bon-Huen-Sao, Wu-Sao x3}{Br\"uckenarm, etc.}
		\WTGalerySlideshowFour{siunimtau/3/9}{siunimtau/3/10}{siunimtau/3/11}{siunimtau/3/6}
		
		wieder bei f) anfangen bis n) dreimal.
	\end{WTSatzTeil}
	\begin{WTSatzTeil}{Djark-Cheung}{Seitlicher Handfl\"achenstoss}
		\WTGalerySlideshowTwo{siunimtau/3/9}{siunimtau/3/20}
		
		asdf.
	\end{WTSatzTeil}
	\begin{WTSatzTeil}{Ching-Cheung}{Gerader Handfl\"achenstoss}
		\WTGalerySlideshowThree{siunimtau/3/21}{siunimtau/3/22}{siunimtau/3/23}
		
		Plus am Ende mit Tan-Sao die Schlusssequenz.
	\end{WTSatzTeil}
\end{WTSatz}

%%%%%%%%%%%%%%%%%%%%%%%%%%%%%%%%%%%%%%%%%%

\begin{WTSatz}{Lange Br\"ucke}% 4. SATZ
	\begin{WTSatzTeil}{Jor-/Yau-Gam-Sao}{Linke und rechte Hinunterdr\"uckende Arme}
		\WTGalerySlideshowThree{siunimtau/4/1}{siunimtau/4/2}{siunimtau/4/3}
		\WTGalerySlideshowTwo{siunimtau/4/4}{siunimtau/4/5}
		
		Asdf
	\end{WTSatzTeil}
	\begin{WTSatzTeil}{Hau-/Chin-Gam-Sao}{Hinten und Vorne Dr\"uckende Arme}
		\WTGalerySlideshowThree{siunimtau/4/6}{siunimtau/4/7}{siunimtau/4/8}
		\WTGalerySlideshowTwo{siunimtau/4/9}{siunimtau/4/10}
		Asdf
	\end{WTSatzTeil}
%		\WTGalerySlideshowFour{siunimtau/4/11}{siunimtau/4/12}{siunimtau/4/13}{siunimtau/4/14}
	\begin{WTSatzTeil}{Shang-Lan-Sao, Shang-Fak-Sao}{Doppelter Riegelarm, Doppelter Handkantenschlag}
		\WTGalerySlideshowThree{siunimtau/4/14}{siunimtau/4/15}{siunimtau/4/16}
		
		Zuerst Lan, dann schauen, dann Fak, dann zurueck Lan mit rechts oben.
		
		ellbogen fuehrt.
	\end{WTSatzTeil}
	\begin{WTSatzTeil}{Shang-Djam-Sao}{Doppelt sinkende Ellb\"ogen}
		\WTGalerySlideshowTwo{siunimtau/4/17}{siunimtau/4/18}
		
		asdf.
	\end{WTSatzTeil}
	\begin{WTSatzTeil}{Shang-Tan/Tok-Sao}{Doppelt XXX}
		\WTGalerySlideshowTwo{siunimtau/4/19}{siunimtau/4/20}
		
		asdf.
	\end{WTSatzTeil}
	\begin{WTSatzTeil}{Shang-Djat-Sao, Shang-Biu-Tze-Sao}{Doppelt ???, Doppelte Fingerstiche}
		\WTGalerySlideshowThree{siunimtau/4/21}{siunimtau/4/22}{siunimtau/4/23}
		
		asdf.
	\end{WTSatzTeil}
	\begin{WTSatzTeil}{Cheung-Kiu-Gam-Sao}{Lange Br\"ucke Gam-Sao}
		\WTGalerySlideshowTwo{siunimtau/4/23}{siunimtau/4/24}
		
		asdf
	\end{WTSatzTeil}
	\begin{WTSatzTeil}{Shang-Tai-Sao}{Doppelter ???}
		\WTGalerySlideshowFour{siunimtau/4/25}{siunimtau/4/26}{siunimtau/4/27}{siunimtau/4/28}
		
		asdf
	\end{WTSatzTeil}
	\begin{WTSatzTeil}{Huen-Sao, Sao-Kuen}{Schlusssequenz}
		
		mit verkehrtem, doppelten huen-sao.
	\end{WTSatzTeil}
\end{WTSatz}

%%%%%%%%%%%%%%%%%%%%%%%%%%%%%%%%%%%%%%%%%%

\begin{WTSatz}{Handfl\"achensto{\ss}}% 5. SATZ
	\begin{WTSatzTeil}{Djark-Cheung}{Seitlicher Handfl\"achenstoss}
		\WTGalerySlideshowThree{siunimtau/5/1}{siunimtau/5/2}{siunimtau/5/3}

		Asdf.
	\end{WTSatzTeil}
	\begin{WTSatzTeil}{Wang-Cheung}{Gerader Handfl\"achenstoss}
		\WTGalerySlideshowThree{siunimtau/5/4}{siunimtau/5/5}{siunimtau/5/6}

		am ende Schlusssequenz.
	\end{WTSatzTeil}
\end{WTSatz}


%%%%%%%%%%%%%%%%%%%%%%%%%%%%%%%%%%%%%%%%%%

\begin{WTSatz}{Ein D malen}% 6. SATZ
	\begin{WTSatzTeil}{Tan-Sao, Djam-Sao}{?}
		\WTGalerySlideshowFour{siunimtau/6/1}{siunimtau/6/2}{siunimtau/6/3}{siunimtau/6/4}
		
		Asdf.
	\end{WTSatzTeil}
	\begin{WTSatzTeil}{Gwat-Sao}{?}
		\WTGalerySlideshowTwo{siunimtau/6/4}{siunimtau/6/5}
		
		asdf.
	\end{WTSatzTeil}
	\begin{WTSatzTeil}{Lau-Sao, Ko-Tan-Sao}{?, Hoher Tan-Sao}
		\WTGalerySlideshowTwo{siunimtau/6/6}{siunimtau/6/7}
		
		Manchmal wird der Ko-Tan-Sao (eine Position) auch als Tok-Sao (eine Bewegung die zu der Position f\"uhrt) bezeichnet.
	\end{WTSatzTeil}
	\begin{WTSatzTeil}{Dai-Cheung}{?}
		\WTGalerySlideshowFour{siunimtau/6/8}{siunimtau/6/9}{siunimtau/6/10}{siunimtau/6/11}
		
		schlusssequenz.
		
		checken ob handflaeche in der mitte ist durch hinzunehmen des zweiten armes:
		\WTGalerySlideshowTwo{siunimtau/6/10}{siunimtau/6/10b}
	\end{WTSatzTeil}
\end{WTSatz}

%%%%%%%%%%%%%%%%%%%%%%%%%%%%%%%%%%%%%%%%%%

\begin{WTSatz}{Schwingen ausbreiten}% 7. SATZ
	\begin{WTSatzTeil}{Bong-Sao}{?}
		\WTGalerySlideshowTwo{siunimtau/7/1}{siunimtau/7/2}
		
		Asdf.
	\end{WTSatzTeil}
	\begin{WTSatzTeil}{Tan-Sao}{?}
		\WTGalerySlideshowTwo{siunimtau/7/2}{siunimtau/7/3}
		
		asdf.
	\end{WTSatzTeil}
	\begin{WTSatzTeil}{Ong-Cheung}{Verkehrter Handfl\"achenstoss}
		\WTGalerySlideshowThree{siunimtau/7/4}{siunimtau/7/5}{siunimtau/7/6}
		
		Schlusssequenz.
	\end{WTSatzTeil}
\end{WTSatz}

%%%%%%%%%%%%%%%%%%%%%%%%%%%%%%%%%%%%%%%%%%

\begin{WTSatz}{Befreiungssatz}% 8. SATZ
	\begin{WTSatzTeil}{Tuet-Sao x3}{?}
	
		Zuerst stechen, bzw Gam-Sao.
		
		\WTGalerySlideshowThree{siunimtau/8/1}{siunimtau/8/2}{siunimtau/8/3}
		
		Befreiung links.
		
		\WTGalerySlideshowThree{siunimtau/8/4}{siunimtau/8/5}{siunimtau/8/6}
		
		Befreiung rechts.
		
		\WTGalerySlideshowThree{siunimtau/8/7}{siunimtau/8/8}{siunimtau/8/3}
		
		Befreiung links und Faust vorbereiten.
		
		\WTGalerySlideshowThree{siunimtau/8/4}{siunimtau/8/5}{siunimtau/8/20}

		asdf foobar asdf foobar asdf foobar asdf foobar asdf foobar asdf.
	\end{WTSatzTeil}
	\begin{WTSatzTeil}{Lin-Wan-Chung-Kuen x3}{Faustst\"o{\ss}e}
		
		Erster.
	
		\WTGalerySlideshowThree{siunimtau/8/20}{siunimtau/8/21}{siunimtau/8/22}
		
		Zweiter.
		
		\WTGalerySlideshowTwo{siunimtau/8/23}{siunimtau/8/24}
		
		asdf foobar asdf foobar asdf foobar asdf foobar asdf foobar asdf.
		% TODO Sao-Sik ?
		
		Schlusssequenz.
	\end{WTSatzTeil}
\end{WTSatz}




%\subsection{Anwendungen zur Siu-Nim-Tau}
%%%%%%%%%%%%%%%%%%%%%%%%%%%%%%%%%%%%%%%%%%
%%%%%%%%%%%%%%%%%%%%%%%%%%%%%%%%%%%%%%%%%%

%Bla bla.

%\subsection{Chum-Kiu}
% ... brueckensuchende form
