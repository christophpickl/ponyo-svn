
%\definecolor{grey}{rgb}{0.9,0.9,0.9} 


% fuer \WTKurzSatz in formen.tex, und \WTEWTOLehrer in ewto.tex
\def\WTXCommonImageTop#1{\vtop{\null\hbox{#1}}}

%%
% table additions
%%%%%%%%%%%%%%%%%%%%%%%%%%%%%%%%%%%%

\newcommand{\WTXcell}[2]{\begin{tabular}[t]{@{}#2@{}}#1\end{tabular}}
\newcommand{\WTcell}[1]{\WTXcell{#1}{l}}%left
\newcommand{\WTcellCentered}[1]{\WTXcell{#1}{c}}


%%
% define custom enumeration style
%%%%%%%%%%%%%%%%%%%%%%%%%%%%%%%%%%%%

\newlist{WTalphenum}{enumerate}{2}
\setlist[WTalphenum,1]{label=\alph*)}
\setlist[WTalphenum,2]{label=\roman*)}

\newenvironment{WTalphenumNarrow}{
\begin{WTalphenum}
  \setlength{\itemsep}{1pt}
  \setlength{\parskip}{2pt}
  \setlength{\parsep}{0pt}
}{\end{WTalphenum}}
	

%%%%%%%%%%%%%%%%%%%%%%%%%%%%%%%%%%%%


\def \quote #1 {``#1''}
\def \quoteit #1 {``\textit{#1}''}

\newenvironment{enumerateNarrow}{
\begin{enumerate}
  \setlength{\itemsep}{1pt}
  \setlength{\parskip}{2pt}
  \setlength{\parsep}{0pt}
}{\end{enumerate}}

\newenvironment{itemizeNarrow}{
\begin{itemize}
  \setlength{\itemsep}{1pt}
  \setlength{\parskip}{2pt}
  \setlength{\parsep}{0pt}
}{\end{itemize}}


\def\WTCommonCite #1#2{
	\begin{adjustwidth}{1cm}{1cm}
	\textit{``#2''}
	\flushright{--- #1}
\end{adjustwidth}
}

	
\newenvironment{WTCommonNoob}
	{
		\definecolor{WTXCommonColoredBoxVariable}{rgb}{1.0,0.3,0.3}
		\begin{WTXCommonColorBoxEnv}{common_noob_alert_sign}{Darauf ist zu achten!}
	} {
		\end{WTXCommonColorBoxEnv}
	}


\newenvironment{WTCommonBegriff}
	{
		\definecolor{WTXCommonColoredBoxVariable}{rgb}{0.96,0.97,0.98}
		\begin{WTXCommonColorBoxEnv}{common_begriff_sign}{Technicus Terminus!}
	} {
		\end{WTXCommonColorBoxEnv}
	}

%\newenvironment{WTCommonPruefen}
%	{
%		\definecolor{WTXCommonColoredBoxVariable}{rgb}{0.80,0.91,0.98} % TODO gscheite farbe noch waehlen fuer Pruefen (irgendwas gelbes, als achtung maessig)
%		\begin{WTXCommonColorBoxEnv}{common_begriff_sign}{Korrektheit \"Uberpr\"ufen!}
%	} {
%		\end{WTXCommonColorBoxEnv}
%	}


\definecolor{WTXCommonColoredBoxVariable}{rgb}{1.0,0.0,0.0} % start with default color REEEED


% =================== INTERNAL

\makeatletter\newenvironment{WTXCommonColoredBox}
	{
		\begin{lrbox}{\@tempboxa}\begin{minipage}[b]{400px}
	} {
   		\end{minipage}\end{lrbox}%\columnwidth statt 400px
   		\colorbox{WTXCommonColoredBoxVariable}{\usebox{\@tempboxa}}
	}\makeatother

%\makeatletter\newenvironment{WTXCommonColoredBox}[1][]{%
%   \begin{lrbox}{\@tempboxa}\begin{minipage}{400px}}{\end{minipage}\end{lrbox}%\columnwidth statt 400px
%   \colorbox[HTML]{FF00FF}{\usebox{\@tempboxa}}
%}\makeatother


\newenvironment{WTXCommonColorBoxEnv}[2]
	{
		\begin{flushleft}
			\begin{tabular}{lc}
				\includegraphics[width=32px]{resources/images/icons/#1}
				&
				\begin{WTXCommonColoredBox}
				\textbf{#2} \vspace{3px} \\
	} {
				\end{WTXCommonColoredBox}
			\end{tabular}
		\end{flushleft}
	}


%\colorbox{red}{Black text on red background}
\def\WTXCommonColorBox #1#2#3#4{
	\begin{flushleft}
		\begin{tabular}{lc}
			\includegraphics[width=32px]{resources/images/icons/#2}
			&
			\colorbox[HTML]{#1} {
				\begin{minipage}[b]{400px}
					\textbf{#3}\\ #4
				\end{minipage}
			}
		\end{tabular}
	\end{flushleft} 
}


%\colorbox{grey}{
%	\parbox[t]{1.0\linewidth}{
%		\printtitle 
%		\vspace*{0.7cm}
%	}
%}