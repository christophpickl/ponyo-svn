

\newenvironment{WTSatz}[1]
	{\WTGaleryResetSlideshowCounter \subsection{#1}}
	{}

\newenvironment{WTSatzTeil}[2]
	{\paragraph{#1} (\textit{#2})}
	{}

\def\WTXFormen_EingangsGraphics#1{\includegraphics[width=2.2cm]{resources/images/eingangsform/#1}}


\renewcommand\chapterillustration{pushing_minimalistisch}
\chapter{Die Form}

Die Formen setzen sich aus einer Reihe von vordefinierten Bewegungsabl\"aufen zusammen, die von jedem WT-Praktizierenden regelm\"a{\ss}ig, aufmerksam und bewusst durchgef\"uhrt werden sollte. Die Art der Ausf\"uhrung hierbei kann gerne variieren: mal schnell/langsam, mal hart/weich, mal auf dem einen/anderen Bein, mit nur einer Hand, etc. Auf jeden Fall sollte nicht durch Hetzerei die Bewegung \textit{verwischt} werden, sondern jeder Punkt noch durchlaufen werden.

Nach l\"angerem \"Uben schleifen sich die Bewegungen ein und sind sofort f\"ur einen abrufbar. Dies gibt einem die M\"oglichkeit sich eine Art Werkzeugkasten zu bauen, aus dem man sich dann bei Bedarf bedient um eine konkrete Technik umzusetzen.

\newpage


\section{Begin und Schluss}
%%%%%%%%%%%%%%%%%%%%%%%%%%%%%%%%%%%%%%%%%%
%%%%%%%%%%%%%%%%%%%%%%%%%%%%%%%%%%%%%%%%%%

Die \textbf{Eingangsform} geht von einem lockeren Stand aus, und f\"uhrt in den sogenannten IRAS\index{IRAS}-Stand, was soviel bedeutet wie: \textit{\textbf{I}nternally \textbf{R}otated \textbf{A}dduction \textbf{S}tance} (oder auch auf Chinesisch \textit{Yee-Gee-Kim-Yeung-Ma} mit w\"ortlicher \"Ubersetzung Schriftzeichen-Zwei-Adduktions-Stand). Hierbei werden die Knie zueinandergepresst wobei die Oberschenkelmuskulator (die Adduktoren; v. lat.: \textit{adducere}, hinf\"uhren, hinziehen) durchgehend gespannt und somit trainiert werden. Dieser Stand versteht sich nur als Trainingsstand zur Verbesserung der Beinmuskulator und ist nicht gedacht zum Einsatz im Ernstfall.

\begin{figure}[htbp]
	\centering
	\begin{tabular}{ccccc}
		\WTXFormen_EingangsGraphics{arm1} & \WTXFormen_EingangsGraphics{arm2} & \WTXFormen_EingangsGraphics{arm3} & \WTXFormen_EingangsGraphics{arm3} & \WTXFormen_EingangsGraphics{arm3} \\
		\WTXFormen_EingangsGraphics{bein1} & \WTXFormen_EingangsGraphics{bein2} & \WTXFormen_EingangsGraphics{bein3} & \WTXFormen_EingangsGraphics{bein4} & \WTXFormen_EingangsGraphics{bein5} \\
		a) & b) & c) & d) & e) \\
	\end{tabular}
\end{figure}

Ausgangspunkt ist ein aufrecher, entspannter Stand a), der Blick ist nach vorne gerichtet, der Kopf sitzt weiter hinten, die Arme h\"angen locker auf der Seite hinunter, die F\"usse sind gschlossen und die H\"ande offen. Nun werden beide H\"ande zu F\"austen gemacht und dann der Seite entlang brusthoch angezogen b), die Fingerkn\"ocheln schauen nicht vor dem K\"orper und die Ellbogen nicht aus ihm heraus, die Unterarme sind parallel zum Boden. Dann werden die Knie gebeugt c) und man setzt sich leicht mit aufrechtem Oberk\"orper in die Hocke. Um den Stand zu verbreitern dreht man die Zehenspitzen so weit wie m\"oglich auf den Fersen auseinander d). Danach folgen die Fersen e) und werden \"uber die Fussmitte aufgemacht sodass beide F\"u{\ss}e ein gleichseitiges Dreieck am Boden formen.

\WTCommonNoob{Man dreht gerne auf den Zehen auseinander was zu einem falschen Stand f\"uhrt.}

\WTGaleryImageTwo{eingangsform/fertig-front}{eingangsform/fertig-seite}{Demonstration des IRAS Stand}

Am Ende gibt es dann eine kurzgehaltene \textbf{Ausgangsform} die wieder zur\"uck in einen lockeren Stand f\"uhrt:

\begin{WTalphenum}
	\item Ausgangspunkt ist der volle IRAS Stand.
	\item Als erstes wird der rechte Fuss gerade gestellt, sodass Zehen und Ferse eine gerade Linie nach vorne bilden.
	\item Nun stellt (noch immer gehockt) der linke Fuss zum rechten parallel Fuss mit kleinem Abstand zu.
	\item Erst jetzt werden die Beine aus der Hocke gestreckt.
	\item Und zu letzt werden die Arme an der Seite locker hinuntergef\"uhrt.
\end{WTalphenum}


\section{Siu-Nim-Tau}
%%%%%%%%%%%%%%%%%%%%%%%%%%%%%%%%%%%%%%%%%%
%%%%%%%%%%%%%%%%%%%%%%%%%%%%%%%%%%%%%%%%%%
% http://everything2.com/title/Sil+Lum+Tao

Siu-Nim-Tau bedeutet so etwas wie \textit{Kleine Idee} oder auch \textit{kleine Ideen Form}. Klein deshalb, da man recht isolierte Bewegungen ausf\"uhrt und sich auf das kleine, die Details konzentriert. So gibt es in dieser Form z.B. keinerlei Wendungen oder Schritte, der Oberk\"orper und die Beine bleiben durchgehend in der selben Position und bewegen sich nicht.

Das Motto der SNT ist \textit{Punkt-Punkt-Klar}. Es sollen jede Stationen (Punkte) der Bewegungen klar ersichtlich erreicht werden. Ein Nehmen von Abk\"urzungen ist also nicht angebracht und verf\"alschen den Ablauf. Weitere Motto die beim \"Uben helfen:

\begin{itemize}
	\item Dr\"ucke deinen Kopf Richtung Himmel und stehe fest am Boden.
	\item Kopf hoch mit horizontalem Blick.
	\item Aufnahmef\"ahiger Brustkorb und aufgerichteter R\"ucken.
	\item H\"ufte gerade und Bauch einziehen
	\item Bei allen Bewegungen der Arme zu beachten: Tiefer Ellbogen und entspannte Schulterhaltung
	\item Wenn eine Armbewegung erfolgt: Schau in die Richtung der Handbewegung.
\end{itemize}

\WTCommonBegriff{Siu-Nim-Tau kann auch geschrieben werden als \textit{Siu-Lin-Tau} (ein bisschen am Anfang \"uben) oder als \textit{Siu-Lam-Tau} (Vergiss nicht, dass der Ursprung des wing chun im Shaolin liegt)}
 

 % TODO es lehrt einem wo seine mitte ist, man lernt wo mein koerper aufhoert; guter stand, sich wurzeln (zehen in den boden) siehe IRAS


\begin{WTSatz}{Schneidende Arme}% 1. SATZ
%%%%%%%%%%%%%%%%%%%%%%%%%%%%%%%%%%%%%%%%%%

	\begin{WTSatzTeil}{Gau-Cha-Tan-Sao}{Gekreuzte Handfl\"ache nach oben Arme}
		\WTGalerySlideshowFour{siunimtau/1-1}{siunimtau/1-2}{siunimtau/1-3}{siunimtau/1-3b}
		
		Als allererstes werden die geschlossenen F\"auste a) ganz ge\"offnet sodass die Handfl\"achen eine gerade Oberfl\"ache bilden b). Danach werden die H\"ande gef\"uhrt von den Ellbogen nach vorne geschoben c) bis dass ein Abstand von einer Faust breit zwischen K\"orper und Ellbogen besteht, und die Fingerspitzen auf H\"ohe der Schulter ist d).
	\end{WTSatzTeil}
	
	\begin{WTSatzTeil}{Gau-Cha-Gan-Sao}{Gekreuzter schneidender Arm}
		\WTGalerySlideshowTwo{siunimtau/1-3}{siunimtau/1-4}
		
		Dann schneiden.
	\end{WTSatzTeil}
	
	\begin{WTSatzTeil}{Kwan-Sao}{?}
		\WTGalerySlideshowThree{siunimtau/1-4}{siunimtau/1-5}{siunimtau/1-6}
		
		Hochdrehen. ACHTUNG NOOB: Gelenke nicht abwinkeln.
	\end{WTSatzTeil}
	
	\begin{WTSatzTeil}{Sau-Kuen}{?}
		\WTGalerySlideshowThree{siunimtau/1-6}{siunimtau/1-7}{siunimtau/1-8}
		
		Zurueckziehen.
	\end{WTSatzTeil}

\end{WTSatz}

%%%%%%%%%%%%%%%%%%%%%%%%%%%%%%%%%%%%%%%%%%

\begin{WTSatz}{Einfacher Fauststo{\ss}}% 2. SATZ
	\begin{WTSatzTeil}{Eindrehen, Zur Mitte, Fauststoss}{?}
		
		Linker und rechter Yat-Gee-Chung-Kuen.
	\end{WTSatzTeil}
	\begin{WTSatzTeil}{Huen-Sao, Sao-Kuen}{?}
		
		asdf.
	\end{WTSatzTeil}
\end{WTSatz}

%%%%%%%%%%%%%%%%%%%%%%%%%%%%%%%%%%%%%%%%%%

\begin{WTSatz}{Dreifache Verehrung Buddhas}% 3. SATZ
	
	\WTCommonBegriff{Der dritte Satz wird im Originalen mit \textit{Sam Bai Fat}\index{Sam Bai Fat} \"ubersetzt; oder auch \textit{Fat Shan} in Kantonesisch}
	
	\begin{WTSatzTeil}{Tan-Sao}{?}
		
		Asdf.
	\end{WTSatzTeil}
	\begin{WTSatzTeil}{Bon-Huen-Sao, Wu-Sao}{?}
		
		Bon = Halber.
	\end{WTSatzTeil}
	\begin{WTSatzTeil}{Fook-Sao, Bon-Huen-Sao, Wu-Sao x3}{?}
		
		asdf.
	\end{WTSatzTeil}
	\begin{WTSatzTeil}{Djark-Cheung, Ching-Cheung, Huen-Sao}{?}
		
		asdf.
	\end{WTSatzTeil}
\end{WTSatz}

%%%%%%%%%%%%%%%%%%%%%%%%%%%%%%%%%%%%%%%%%%

\begin{WTSatz}{Lange Br\"ucke}% 4. SATZ
	\begin{WTSatzTeil}{Jor-/Yau-Gam-Sao}{?}
		
		Jor = Links, Yau = Rechts.
	\end{WTSatzTeil}
	\begin{WTSatzTeil}{Hau-/Chin-Gam-Sao}{?}
		
		Hau = Hinter, Chin = Vorder
	\end{WTSatzTeil}
	\begin{WTSatzTeil}{Shang-Lan-Sao, Shang-Fak-Sao}{?}
		
		Shang = Doppelt.
	\end{WTSatzTeil}
	\begin{WTSatzTeil}{Shang-Lan-Sao, Shang-Djam-Sao}{?}
		
		asdf.
	\end{WTSatzTeil}
	\begin{WTSatzTeil}{Shang-Tan/Tok-Sao, Shang-Djat-Sao, Shang-Biu-Tze-Sao}{?}
		
		asdf.
	\end{WTSatzTeil}
	\begin{WTSatzTeil}{Cheung-Kiu-Gam-Sao, Shang-Tai-Sao, Sao-Kuen}{?}
		
		Cheung-Kiu = Lange Br\"ucke.
	\end{WTSatzTeil}
\end{WTSatz}

%%%%%%%%%%%%%%%%%%%%%%%%%%%%%%%%%%%%%%%%%%

\begin{WTSatz}{Handfl\"achensto{\ss}}% 5. SATZ
	\begin{WTSatzTeil}{Djark-Cheung, Wang-Cheung}{?}
		
		Asdf.
	\end{WTSatzTeil}
	\begin{WTSatzTeil}{Huen-Sao, Sao-Kuen}{?}
		
		asdf.
	\end{WTSatzTeil}
\end{WTSatz}


%%%%%%%%%%%%%%%%%%%%%%%%%%%%%%%%%%%%%%%%%%

\begin{WTSatz}{Ein D malen}% 6. SATZ
	\begin{WTSatzTeil}{Tan-Sao, Djam-Sao}{?}
		
		Asdf.
	\end{WTSatzTeil}
	\begin{WTSatzTeil}{Gwat-Sao}{?}
		
		asdf.
	\end{WTSatzTeil}
	\begin{WTSatzTeil}{Lau-Sao, Ko-Tan-Sao}{?, Hoher Tan-Sao}
		
		Manchmal wird der Ko-Tan-Sao (eine Position) auch als Tok-Sao (eine Bewegung die zu der Position f\"uhrt) bezeichnet.
	\end{WTSatzTeil}
	\begin{WTSatzTeil}{Dai-Cheung}{?}
		
		asdf.
	\end{WTSatzTeil}
	\begin{WTSatzTeil}{Huen-Sao, Sao-Kuen}{?}
		
		asdf.
	\end{WTSatzTeil}
\end{WTSatz}

%%%%%%%%%%%%%%%%%%%%%%%%%%%%%%%%%%%%%%%%%%

\begin{WTSatz}{Schwingen ausbreiten}% 7. SATZ
	\begin{WTSatzTeil}{Bong-Sao}{?}
		
		Asdf.
	\end{WTSatzTeil}
	\begin{WTSatzTeil}{Tan-Sao}{?}
		
		asdf.
	\end{WTSatzTeil}
	\begin{WTSatzTeil}{Ong-Cheung}{?}
		
		asdf.
	\end{WTSatzTeil}
	\begin{WTSatzTeil}{Tan-Sao, Huen-Sao, Sao-Kuen}{?}
		
		asdf.
	\end{WTSatzTeil}
\end{WTSatz}

%%%%%%%%%%%%%%%%%%%%%%%%%%%%%%%%%%%%%%%%%%

\begin{WTSatz}{Befreiungssatz}% 8. SATZ
	\begin{WTSatzTeil}{Tuet-Sao x3}{?}
		
		Asdf.
	\end{WTSatzTeil}
	\begin{WTSatzTeil}{Lin-Wan-Chung-Kuen x3}{?}
		
		asdf.
	\end{WTSatzTeil}
	\begin{WTSatzTeil}{Huen-Sao, Sao-Sik}{?}
		
		asdf.
	\end{WTSatzTeil}
\end{WTSatz}




%\subsection{Anwendungen zur Siu-Nim-Tau}
%%%%%%%%%%%%%%%%%%%%%%%%%%%%%%%%%%%%%%%%%%
%%%%%%%%%%%%%%%%%%%%%%%%%%%%%%%%%%%%%%%%%%

%Bla bla.

%\subsection{Chum-Kiu}
% ... brueckensuchende form
