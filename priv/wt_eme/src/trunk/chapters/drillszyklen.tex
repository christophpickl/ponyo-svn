
% XXXXXXXXXXXXXXXXXXXXXXXXXXXXXXXXXXXXXXXXXXXXXXXXXXXXXXXXXXXXXXXXXXXXXXXX

\renewcommand\chapterillustration{pushing_minimalistisch}
\chapter{Drills und Zyklen}


% =============================================================================
\section{Solo\"ubungen}
% =============================================================================

siehe auch ewto programme (gleizeitigkeitenzyklus)

% TODO evtl referenzieren auf fauststoesse gegen wand mit polster dranlegen, darf nicht runterfallen

% TODO tai chi uebungen
% - gewichtsverlagerung pruefen mit vielen waagen am boden (weiter auseinander dass schwerer)
% - auf einem bein stehen (tiefer und laenger dass schwerer)

% XXXXXXXXXXXXXXXXXXXXXXXXXXXXXXXXXXXXXXXXXXXXXXXXXXXXXXXXXXXXXXXXXXXXXXXXXXXX
\subsection{Schrittzyklus}

\begin{enumerate}
	\item biu tse kreise am boden
	\item ueblicher zirkelschritt
	\item ...
	\item 45 grad wenden, 90 grad wenden, ueberwenden
	\item zieh-wende schrit
	\item Z-Z schritt
	\item kreuzschritt
	\item ...
	\item stampf schritt
	\item 180 grad drehung
\end{enumerate}

% XXXXXXXXXXXXXXXXXXXXXXXXXXXXXXXXXXXXXXXXXXXXXXXXXXXXXXXXXXXXXXXXXXXXXXXXXXXX
\subsection{Gleichzeitigkeitenzyklus}

aus dem 3. ewto programm.
hat zwei teile: 1-4 immer gleich, 5-8 gibts variationen.
kann auch mit partner gemacht werden (als zusaetzliche druck-reflex uebung).

% =============================================================================
\section{Partner\"ubungen}
% =============================================================================

eigentlich gesamte chi sao sektionen.

siehe auch ewto programme (krabbengang)


% XXXXXXXXXXXXXXXXXXXXXXXXXXXXXXXXXXXXXXXXXXXXXXXXXXXXXXXXXXXXXXXXXXXXXXXXXXXX
\subsection{Gleichzeitigkeitenzyklus}

aus dem 3. ewto programm.
ist auch gut fuer druck/kontakt-reflex.
kann man auch alleine machen, halt ohne druck nur fuer gleichzeitigkeit.

% XXXXXXXXXXXXXXXXXXXXXXXXXXXXXXXXXXXXXXXXXXXXXXXXXXXXXXXXXXXXXXXXXXXXXXXXXXXX
\subsection{Schwingerzyklus}

aus dem 4. ewto programm

% XXXXXXXXXXXXXXXXXXXXXXXXXXXXXXXXXXXXXXXXXXXXXXXXXXXXXXXXXXXXXXXXXXXXXXXXXXXX
\subsection{Ellbogenzyklus}

aus dem 5. programm
verschiedene arten zu uebergreifen
mittig der erstkontakt, sonst rutscht was rueber
traditional, oder modern mit gewicht vorne


% XXXXXXXXXXXXXXXXXXXXXXXXXXXXXXXXXXXXXXXXXXXXXXXXXXXXXXXXXXXXXXXXXXXXXXXXXXXX
\subsection{XX Lie Man Kueh XX}

alias aussen-aussen.

% XXXXXXXXXXXXXXXXXXXXXXXXXXXXXXXXXXXXXXXXXXXXXXXXXXXXXXXXXXXXXXXXXXXXXXXXXXXX
\subsection{12er Angriff a la Thomas}

\begin{enumerate}
	\item von aussen, aeussere Hand vorne mit Faustschlag gerade vor, wenn haende normal hoch
	\item von in-side, links aufkeilen, dann zum tan und rechts schlagen, wenn haende weiter offen
	\item von aussen, innere Hand vorne mit Faustschlag gerade vor, wenn haende normal hoch
	\item von in-side, selbe hand wie vorhin such sofort kontakt zum pak, wenn haende weiter zusammen
	\item sofort Tok, wenn extrem hohe Armhaltung (gegen thai? kick boxer)
	\item schlaengeln und fesseln, wenn engen haenden
	\item diagonal schlagen, einfacher 3er angriff, wenn haende ein bissi hoeher
	\item mit doppelaufkeilen zu lap, wenn haende ein bissi hoeher
	\item tan tai cheung, wenn haende weiter oben
	\item normaler pak sao, wenn haende weiter unten
	\item erweiterter 7er, also 3er angriff mit zusaetzlicher blockade, wenn haende ein bissi hoeher
	\item tan fauststoss, wenn haende weiter unten
\end{enumerate}

% XXXXXXXXXXXXXXXXXXXXXXXXXXXXXXXXXXXXXXXXXXXXXXXXXXXXXXXXXXXXXXXXXXXXXXXXXXXX
\subsection{Tan-Pak-Lap-Bong Drill}


% =============================================================================
\section{Angriffssequenzen}
% =============================================================================

% XXXXXXXXXXXXXXXXXXXXXXXXXXXXXXXXXXXXXXXXXXXXXXXXXXXXXXXXXXXXXXXXXXXXXXXXXXXX
\subsection{4er Angriff a la Gerhard}

% XXXXXXXXXXXXXXXXXXXXXXXXXXXXXXXXXXXXXXXXXXXXXXXXXXXXXXXXXXXXXXXXXXXXXXXXXXXX
\subsection{pak tiefer-punch a la ado}
mit pak und tiefer stoss a la ado (dann mit ellbogen runter und von hinten aufmachen)
auch gegenwehr a la holzpuppe vom robert moeglich, mit runter gehn und aufkeilen in lee-man-kue drill

