\newenvironment{WTPosition}[2]
	{
		\subsubsection{#1}
		\textit{#2}
		\\
	}
	{}
	

\renewcommand\chapterillustration{six-computers-chips-circuits}
\chapter{Positionen \& Bewegungen}


\WTCommonCite{Lao-Tse}{Ein ur cooles \\ Zitat vom Lao Tse.\\ Ein ur cooles Zitat vom Lao Tse. Ein ur cooles Zitat vom Lao Tse. Ein ur cooles Zitat vom Lao Tse. Ein ur cooles Zitat vom Lao Tse. Ein ur cooles Zitat vom Lao Tse. Ein ur cooles Zitat vom Lao Tse. Ein ur cooles Zitat vom Lao Tse.}

Bla bla. Wozu.


\newpage



\section{Sechs Grundpositionen}

\begin{WTPosition}{Tan-Sao}{Handfl\"ache oben Hand}
	
	Ist eine absorbierende Hand. Bei der Form Ellbogen innen, sonst Situation anpassen. Flache Hand. Daumen dran. Handfl\"ache parallel zum Boden (oder auch Verl\"angerung zum Unterarm).
	
	%\WTGaleryImage{positionen/tan_sao-front}{9cm}{Tan-Sao Demonstration frontal}
	\WTGaleryImageTwo{positionen/tan_sao-front}{positionen/tan_sao-seite}{Tan-Sao Demonstration frontal und seitlich}
\end{WTPosition}


\section{Zw\"olf Grundbewegungen}

% bong sao	schwing/fluegel arm
%fook sao	~hook sao; controlling/cover hand
%man sao	suchende hand
% wu sao		schuetzende hand
% tan sao		handflaeche oben hand; absorbing/dispersing hand
% kau sao	~klapp sao; detaining hand
%
% jam sao	sinkender ellbogen
% gaun sao	schneidende hand; cultivating arm
% jut sao		choking hand
% huen sao	zirkelnde hand
% lap sao		ziehende hand
% pak sao	klatschende hand; slapping
% tok sao		hebende hand; lifting hand
% lan sao		riegelarm; bar/barring
% ? tie sao		uplifting hand
% ? jip sao		receiving hand
% gum sao	pressing hand
% biu sao		darting hand
% lau sao		? schoepfende hand?

\subsection{Handfl\"achenst\"o{\ss}e}

Djark-Cheung - Seitliche Handfl\"ache (bis auf Schulterbreite; satz 3+5);
?Ching-Cheung - Gerade Handfl\"ache (satz 5);
Wang-Cheung - Liegende Handfl\"ache (satz 5);
Dai-Cheung - Liegende Handfl\"ache zu den kurzen Rippen (satz 6)
Ong-Cheung - Verkehrte Handfl\"ache (satz 7)